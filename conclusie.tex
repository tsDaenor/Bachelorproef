\chapter{Conclusie}
\label{ch:conclusie}

% TODO: Trek een duidelijke conclusie, in de vorm van een antwoord op de
% onderzoeksvra(a)g(en). Reflecteer kritisch over het resultaat. Zijn er
% zaken die nog niet duidelijk zijn? Heeft het ondezoek geleid tot nieuwe
% vragen die uitnodigen tot verder onderzoek?

Er zijn sinds Docker uitgekomen is heel wat kritieken geweest over de werking van Docker. Zowel gerechtvaardigde als ongerechtvaardigde. Docker biedt ons een tool die flexibeler en complexer is dan het gebruik van een gewone virtuele machine. Er zijn hier wat foute opvattingen over Docker geweest in verband met vooral security maar ook op vlak van managebility. Er werden verwachtingen gesteld dat er met Docker niet meer extra moest beveiligd worden, dat op welke manier we onze containers willen gebruiken, ze automatisch veilig en managable zijn. Op het moment dat wat we normaal op een virtuele machine zouden draaien, gaan containerizen moeten we de beveiliging gewoon op een andere plaats voorzien. Als we gebruik maken van virtuele machines moet er nog altijd gebruik gemaakt worden van beveiliging van het netwerk van de virtuele machines. Daarnaast werd ook elke virtuele machine op een andere manier beveiligd. Deze best practices die we vroeger toepasten op de virtuele machines, mogen we nu niet vergeten bij het gebruik van Docker. Elke Docker container moet nog altijd beveiligd worden naargelang wat we willen uitvoeren op de container. Als we Docker gebruiken maken we gebruik van een flexibele, complexe en krachtige tool. zoals Voltaire zei, "With great power comes great responibility". Docker is een relatief nieuwe tool waarvoor een actieve community is die problemen die opkomen zo snel mogelijk oplossen. Als we onze containers op een virtuele machine uitvoeren en zowel de containers als virtuele machine correct beveiligen en up to date houden, is er zeker een veilige plaats voor Docker in een serveromgeving.

Door de extra flexibiliteit die Docker aanbied, wordt het moelijker om onze containers te managen. Er moet kunnen afgestapt worden van het minutieus controleren van containers. De containers moeten zo gemaakt worden dat er zonder veel menselijke tussenkomst langdurig gebruik van gemaakt kan worden. Er kunnen tools voor het managen van Docker containers zowel voor productie als development gebruikt worden. Bij het gebruik van Docker containers wordt logging en monitoring heel wat belangrijker. Wanneer we onze containers meer willen "loslaten" moet er een systeem bestaan die ervoor instaat ons te waarschuwen als er iets misloopt met onze containers.

Docker bied ons een extra laag van virtualisatie aan. Hierdoor mogen er niet dezelfde eisen stellen aan Docker als dat we doen bij virtuele machines. Beide hebben hun plaats op de markt, en hebben de bedoeling om met elkaar samen te werken en niet elkaar vervangen. Hierbij moeten we rekening houden als we wat vroeger op verschillende virtuele machines zou uitgevoerd worden nu laten uitvoeren op één virtuele machine met Docker containers. We een deel van de controle over deze machines verliezen, maar veel meer flexibiliteit verkrijgen. Daarom mag er niet vergeten worden om de veiligheids elementen die vroeger bestonden om de aparte virtuele machines te beveiligen ook worden meegebracht naar het nieuwe systeem met Docker containers.