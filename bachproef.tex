%%========================================================================
%% LaTeX sjabloon voor stage/projectrapport of bachelorproef
%%  HoGent Bedrijf en Organisatie
%%========================================================================

%%========================================================================
%% Preamble
%%========================================================================

\documentclass[pdftex,a4paper,12pt,twoside]{report}

% XXX: Let op: dit sjabloon is gemaakt om dubbelzijdig af te drukken
% Voor enkelzijdig, verwijder ``twoside'' hierboven.

%%---------- Extra functionaliteit ---------------------------------------

\usepackage[utf8]{inputenc}  % Accenten gebruiken in tekst (vb. é ipv \'e)
\usepackage{amsfonts}        % AMS math packages: extra wiskundige
\usepackage{amsmath}         %   symbolen (o.a. getallen-
\usepackage{amssymb}         %   verzamelingen N, R, Z, Q, etc.)
\usepackage[dutch]{babel}    % Taalinstellingen: woordsplitsingen,
                             %  commando's voor speciale karakters
                             %  ("dutch" voor NL)
\usepackage{eurosym}         % Euro-symbool €
\usepackage{geometry}
\usepackage{graphicx}        % Invoegen van tekeningen
\usepackage[pdftex,bookmarks=true]{hyperref}
                             % PDF krijgt klikbare links & verwijzingen,
                             %  inhoudstafel
\usepackage{listings}        % Broncode mooi opmaken
\usepackage{multirow}        % Tekst over verschillende cellen in tabellen
\usepackage{rotating}        % Tabellen en figuren roteren
\usepackage{natbib}          % Betere bibliografiestijlen
\usepackage{fancyhdr}        % Pagina-opmaak met hoofd- en voettekst

\usepackage[T1]{fontenc}     % Ivm lettertypes
\usepackage{lmodern}
\usepackage{textcomp}

\usepackage{lipsum}          % Voor vultekst (lorem ipsum)

\usepackage{booktabs}

\usepackage{xcolor}
%%---------- Layout ------------------------------------------------------

% hoofdingen, enz.
\pagestyle{fancy}
% enkel hoofdstuktitel in hoofding, geen sectietitel (vermijd overlap)
\renewcommand{\sectionmark}[1]{}

% lijn, wordt gebruikt in titelpagina
\newcommand{\HRule}{\rule{\linewidth}{0.5mm}}

% Leeg blad
\newcommand{\emptypage}{
\newpage
\thispagestyle{empty}
\mbox{}
\newpage
}

% Gebruik een schreefloos lettertype ipv het "oubollig" uitziende
% Computer Modern
\renewcommand{\familydefault}{\sfdefault}

% Commando voor invoegen Java-broncodebestanden (dank aan Niels Corneille)
% Gebruik: \codefragment{source/MijnKlasse.java}{Uitleg bij de code}
\newcommand{\codefragment}[2]{ \lstset{%
  language=java,
  breaklines=true,
  float=th,
  caption={#2},
  basicstyle=\scriptsize,
  frame=single,
  extendedchars=\true
}

\lstinputlisting{#1}}

% Opmaak van de paragrafen
\setlength{\parskip}{1em}

\definecolor{codegreen}{rgb}{0,0.6,0}
\definecolor{codegray}{rgb}{0.5,0.5,0.5}
\definecolor{codepurple}{rgb}{0.58,0,0.82}
\definecolor{backcolour}{rgb}{0.95,0.95,0.92}

\lstdefinestyle{configstyle}
{
	backgroundcolor=\color{backcolour},   
	commentstyle=\color{codegreen},
	keywordstyle=\color{red},
	numberstyle=\tiny\color{codegray},
	stringstyle=\color{codepurple},
	basicstyle=\footnotesize,
	breakatwhitespace=false,         
	breaklines=true,                 
	captionpos=b,                    
	keepspaces=true,                 
	numbers=left,                    
	numbersep=5pt,                  
	showspaces=false,                
	showstringspaces=false,
	showtabs=false,                  
	tabsize=2
}


%%---------- Documenteigenschappen ---------------------------------------
%% Vul dit aan met je eigen info:

% Je eigen naam
\newcommand{\student}{Tomas Vercautter}

% De naam van je lector, begeleider, promotor
\newcommand{\promotor}{Bert Van Vreckem}

% De naam van je co-promotor
\newcommand{\copromotor}{Bert Van Vreckem}

% Indien je bachelorproef in opdracht van een bedrijf of organisatie
% geschreven is, geef je hier de naam.
\newcommand{\instelling}{---}

% De titel van het rapport/bachelorproef
\newcommand{\titel}{Security en Managebility van Docker containers}

% Datum van indienen
\newcommand{\datum}{27 mei 2016}

% Faculteit
\newcommand{\faculteit}{Faculteit Bedrijf en Organisatie}

% Soort rapport
\newcommand{\rapporttype}{Scriptie voorgedragen tot het bekomen van de graad van\\Bachelor in de toegepaste informatica}

% Academiejaar
\newcommand{\academiejaar}{2015-2016}

% Examenperiode
%  - 1e semester = 1e examenperiode
%  - 2e semester = 2e examenperiode
%  - tweede zit = 3e examenperiode
\newcommand{\examenperiode}{Tweede examenperiode}

%%========================================================================
%% Inhoud document
%%========================================================================

\begin{document}

%%---------- Front matter ------------------------------------------------
%% Het voorblad - Hier moet je in principe niets wijzigen.

\begin{titlepage}
  \newgeometry{top=2cm,bottom=1.5cm,left=1.5cm,right=1.5cm}
  \begin{center}

    \begingroup
    \rmfamily
    \includegraphics[width=2.5cm]{img/HG-beeldmerk-woordmerk}\\[.5cm]
    \faculteit\\[3cm]
    \titel
    \vfill
    \student\\[3.5cm]
    \rapporttype\\[2cm]
    Promotor:\\
    \promotor\\
    Co-promotor:\\
    \copromotor\\[2.5cm]
    Instelling: \instelling\\[.5cm]
    Academiejaar: \academiejaar\\[.5cm]
    \examenperiode
    \endgroup

  \end{center}
  \restoregeometry
\end{titlepage}

% Schutblad

\emptypage


\begin{titlepage}
  \newgeometry{top=5.35cm,bottom=1.5cm,left=1.5cm,right=1.5cm}
  \begin{center}

    \begingroup
    \rmfamily
    \faculteit\\[3cm]
    \titel
    \vfill
    \student\\[3.5cm]
    \rapporttype\\[2cm]
    Promotor:\\
    \promotor\\
    Co-promotor:\\
    \copromotor\\[2.5cm]
    Instelling: \instelling\\[.5cm]
    Academiejaar: \academiejaar\\[.5cm]
    \examenperiode
    \endgroup

  \end{center}
  \restoregeometry
\end{titlepage}


\begin{abstract}
Sinds het uitkomen van Docker zijn er heel wat vragen geweest in verband met de security en managebility van Docker containers. Heel wat mensen zijn gestopt met het gebruik van Docker door een probleem op een van deze twee vlakken.In dit onderzoek wordt er gezocht naar oplossingen om deze problemen weg te werken. Bij security is er standaard heel wat in orde. Maar er zijn ook wel punten waar er verbetering mogelijk is met enkele simpele toevoegingen. Een element dat veel over het hoofd gekeken, wordt is het beveiligen van de Docker images waar we onze containers op baseren. Bij het managen van Docker containers moet er vooral rekening gehouden worden met het feit dat Docker containers veel flexibeler en vluchtiger zijn dan virtuele machines. Uit het onderzoek kunnen we concluderen dat heel wat problemen met Docker gebaseerd zijn op een verkeerde verwachting van wat Docker kan doen. Docker is een flexibele, krachtige en complexe tool die ook op die manier moet benaderd worden voor het beveiligen en managen ervan.
\end{abstract}

\chapter*{Voorwoord}
\label{ch:voorwoord}

% TODO: Vergeet ook niet te bedanken wie je geholpen/gesteund/... heeft

In deze bachelorproef wordt er van uitgegaan dat de lezer een basis kennis heeft van docker en linux. De belangrijkste elementen van docker die men best vo


\tableofcontents

% Als je een lijst van afkortingen of termen wil toevoegen, dan hoort die
% hier thuis. Gebruik bijvoorbeeld de ``glossaries'' package.

%%---------- Kern --------------------------------------------------------


\chapter{Inleiding}
\label{ch:inleiding}

Het gebruik van docker containers is een manier van virtualiseren van services. Hierbij worden services in een 

In deze bachelorproef zal ik onderzoek doen naar de security en managebility van Docker containers en deze vergelijken met de security en managebility van het draaien van de services rechtstreeks op de virtuele machine.

Deze bachelorproef is onderverdeeld in drie grote delen. In de eerste plaats bespreek ik de security zelf. In een tweede deel zal ik het hebben over de managebility van Docker containers. En als laatste onderdeel bespreek ik logging en monitoring bij Docker containers.

\section{Probleemstelling en context}
\label{sec:onderzoeksvragen}

% TODO: Wees zo concreet mogelijk bij het formuleren van je
% onderzoeksvra(a)g(en). Een onderzoeksvraag is trouwens iets waar nog
% niemand op dit moment een antwoord heeft (voor zover je kan nagaan).

Docker is een nieuwe speler op de virtualisatiemarkt die in korte tijd een substantieel marktaandeel heeft ingenomen. Hiervoor hebben ze Linux LXC containers gebruiksvriendelijker gemaakt. Dit type container ligt ergens in het midden tussen een chroot jail (isoleren van een proces van het rest van het systeem) en een volledige virtuele machine. Sinds Docker is uitgekomen voor het grote publiek zijn er altijd al vragen geweest naar in hoeverre deze vorm van virtualisatie veilig genoeg was om in productie te gebruiken. Als we in Google iets zoeken rond problemen met Docker of mensen die gestopt zijn met Docker te gebruiken, vinden we heel wat posts met als titel iets in de zin van "Why I stopped using Docker". De meeste mensen stoppen met het gebruik van Docker omdat ze problemen vinden met ofwel de beveiliging van de container omgeving, ofwel hebben ze problemen met het managen van hun containers. Deze kritieken of problemen met Docker zijn er al sinds de eerste uitgave van Docker en blijven nu en dan de kop op steken.  

\section{Doelstelling en onderzoeksvraag/-vragen}

De doelstelling van deze bachelor proef bestaat uit het ophelderen van kritieken op vlak van de managebility en beveiliging van deze containers. Hiervoor gaan we eerst op zoek naar welke kritiekpunten er te vinden zijn. er zal zo veel mogelijk gekijken worden naar recentere kritieken omdat Docker snel geëvolueerd is, kunnen kritieken van twee maanden geleden al verouderd zijn. Daarna zullen we in de eerste plaats kijken of deze kritiek gerechtvaardigd is. Indien dit het geval is, zal er onderzocht worden hoe we deze kritiekpunten mogelijks kunnen wegwerken met bepaalde zelf toepasbare technieken of best-practices. Of zijn er elementen die zullen moeten geïmplementeerd worden door de Docker community zelf?

Daarnaast zijn er veel mensen die van Docker containers een veiligere en meer handelbare ervaring verwachten dan als ze met virtuele machines werken. Dus zal er in dit onderzoek ook gekeken worden naar hoe Docker containers op minstens hetzelfde niveau gekregen kunnen worden als een klassieke virtuele machine, zowel qua beheersbaarheid als op beveiligings vlak

\chapter{Methodologie}
\label{ch:methodologie}

% TODO: Hoe ben je te werk gegaan? Verdeel je onderzoek in grote fasen, en
% licht in elke fase toe welke stappen je gevolgd hebt. Verantwoord waarom je
% op deze manier te werk gegaan bent. Je moet kunnen aantonen dat je de best
% mogelijke manier toegepast hebt om een antwoord te vinden op de
% onderzoeksvraag.
Een eerste stap in het onderzoek naar problemen zowel op vlak van security als van managebility bij Docker containers was mezelf wegwijs maken in de wereld van docker. Hiervoor heb ik twee maanden lang elke dag docker intensief gebruikt. Hierbij heb ik een volledige omgeving opgezet waarop applicaties gedeployed werden. Hierdoor kon ik zelf een beeld krijgen van de problemen die andere mensen hebben met Docker en kon ik ook de eventuele oplossingen zelf testen.

Tijdens het opzetten van de omgeving ben ik ook op zoek gegaan naar eventuele kritieken die op het internet te vinden waren en hoe men deze kritieken staafden. In de eerste plaats ging ik op zoek naar eventuele problemen in verband met beveiliging en wanneer die onstonden. Daarnaast ga ik ook op zoek naar problemen in verband met monitoring/managebility. Na het intensief gebruiken van docker en het onderzoek naar problemen had ik een beter zicht welke problemen er nu effectief bestaan in verband met docker container en of deze terecht zijn. Maar ook welke problemen er al waren opgelost.

Na het vinden van deze kritieken moet er gekeken worden of deze kritieken ook voorkomen bij virtuele machines. Meerbepaald of er vroeger op dit vlak ook problemen waren met de virtuele oplossingen en indien zo hoe werd dit opgelost? Ook heb ik gekeken naar hoe de beveiliging en managebility van gewone virtuele machines wordt aangepakt. Dit zorgt er voor dat ik een goed idee heb van waar andere systemen problemen mee hadden en hoe dit verholpen werd/word.

Hierbij kan ik dan de gevonden problemen zowel voor managebility als security bij Dokcer containers gaan vergelijken met de probleempunten bij virtuele machines.  Zijn er elementen die we kunnen oplossen op dezelfde of een gelijkaardige manier als bij virtuele machines. En zijn er dingen die we nog niet hebben tegengekomen bij virtuele machines maar nu wel moeten opgelost worden met docker?

Als laatste ga ik dan kijken hoe we deze kritiekpunten kunnen verhelpen. Meer bepaald zijn er tools, technieken of best-practices die we kunnen gebruiken om dit te verhelpen of zijn er dingen die we niet zelf kunnen oplossen maar moeten opgelost worden door de docker defs?
\part{Corpus}
\label{ch:corpus}

\chapter{Security}

Op vlak van security is er gestart van de CIS Docker 1.6 Benchmark geschreven door \cite{PravinGoyal2015}. Hierin is in samenwerking met de "Center for Internet Security", onderzoek gedaan naar de beveiliging van Docker containers. Dit onderzoek is onderverdeeld in zes grote onderdelen namelijk Host configuration, Docker daemon configuration, Docker daemon configuration files, Container images and build file, Container runtime en Docker security operations. Hierbij wordt er gekeken welke van deze elementen standaard in Docker worden uitgevoerd. In dit hoofdstuk zullen dezelfde zes onderdelen behandeld en vergeleken worden. Hierbij zal er daarnaast gekeken worden hoe we deze beveiliging aanpakken wanneer onze services in een Docker container uitgevoerd worden, tegenover wanneer deze native in onze virtuele machine draaien.


Hierbij zijn ook telkens deze methodes voor Docker containers uitgetest om te kijken hoe makkelijk deze kunnen geïmplementeerd worden in een nieuwe omgevinge of bestaande omgeving. Een grote hulp hiervoor was de github repository "docker-bench-security" van Docker een grote hulp. Dit is een script die alle elementen die worden aangehaald in CIS Docker 1.6 Benchmark automatisch getest en feedback over gegeven.

\section{Host Configuration}

Een van de belangrijkste onderdelen in het beveiligen van de Docker stack, zijnde de combinatie van de docker host machine en de docker containers, is het beveiligen van de host machine waarop de Docker containers zich zullen bevinden. Dit heeft twe grote onderdelen, namelijk het up-to-date houden van het systeem en het juist configureren van het systeem.

\subsection{Up-to-date houden van het systeem}

Om de veiligheid van een systeem te kunnen garanderen moet er ten allen tijde gezorgd worden dat het systeem up-to-date blijft. Dit zorgt ervoor dat veiligheidsgebreken die gekend en gepatcht zijn ook op het systeem beveiligd zijn. Doordat we Docker gebruiken komt hier nog een extra laag bij. Bij het Updaten van een machine waarop de services rechtstreeks draaien moeten enkel het host systeem, de kernel van het host systeem en de services up to date blijven. Doordat Docker een kernel deelt met het hostsysteem wordt het up to date houden van de kernel belangrijker. Naast het onderhouden van de voorvernoemde elementen moet er nu, doordat we Docker gebruiken, nog met twee extra dingen rekening gehouden worden. Enerzijds moet Docker zelf nu ook bijgewerkt worden, maar een onderdeel dat soms vergeten wordt, is dat elke container nu ook zijn eigen besturingssysteem gebruikt. Hierover wordt er verder bij het beveiligen van Container images meer uitleg over gegeven.

% TODO dev tools? and explanaition about stable releases

\subsection{Configureren van het systeem}

Bij het configureren van het hostysteem wordt net zoals bij het up to date houden van het systeem dezelfde stappen van een native applicatie op een hostysteem configureren gevolgd. Maar door de extra laag komen daar nog extra stappen bij. Bij een systeem waar Docker op draait, wordt er aangeraden om een aparte partitie te creëren voor de containers. Dit is een simpele operatie waardoor alle Docker gerelateerde bestanden zich niet meer tussen de bestanden van het hostsysteem bevinden.

% TODO https://docs.docker.com/articles/security/#docker-daemon-attack-surface

Daarnaast moet ook onder controle gehouden worden wie toegang krijgt tot de Docker daemon. De Docker daemon heeft 'root' access, gebruikers die worden toegevoegd aan de 'docker' usergroup en geen 'root' privileges hebben kunnen hierdoor 'root' access verkrijgen tot het hostsysteem. Men kan mappen sharen tussen het host systeem en een guest container en zo toegang verkrijgen tot de gemounte mappen. Doordat containers standaard altijd runnen als 'root' heeft de container, als de '/' map gemount is op deze container, ongelimiteerde toegang tot het volledige host systeem. Hierdoor kunnen zonder restricties aanpassingen aan het hostysteem gedaan worden. Dit betekent dat een gebruiker die toegang heeft tot de Docker daemon verhoogde rechten kan verkrijgen door gewoon een container op te starten.

Zoals bij een virtuele machine waarop de services rechtstreeks worden uitgevoerd, wordt ook bij een systeem waar Docker containers op draaien aangeraden overbodige services uit te schakelen of te verwijderen. Moest een gebruiker door een van deze services toegang krijgen tot het hostsysteem, zou hij makkelijker 'root' rechten kunnen verkrijgen volgens de manier beschreven in de vorige paragraaf.


\section{Docker daemon configuration}

Bij het beveiligen van de docker stack is ook de configuratie van de docker daemon een belangrijk element. Dit heeft twee grote onderdelen. Enerzijds de configuratie van de docker daemon zelf, dit houd in dat we de daemon zelf zo gaan configureren dat deze zo veilig en robuust mogelijk wordt. Aan de andere kant moeten we ook zorgen dat de configuratie files zelf de juiste file permissies en owners hebben. Hierbij is de vergelijking met een virtuele machine, waar de services rechtstreeks op draaien, moeilijk aangezien er hier geen docker daemon op te vinden is. Omdat het configureren van de Docker daemon dingen verandert doe voor elke container toegepast worden kunnen we het wel vergelijken met hoe de inviduele services globaal geconfigureerd worden.

% TODO find information about docker binding on IP/Port or UNIX socket
% https://docs.docker.com/articles/basics/#bind-docker-to-another-hostport-or-a-unix-socket

\subsection{Globale configuratie}

% TODO find info about Docker switch from lxs to libcontainer

% lxc
% logging level
% storage driver

\subsection{netwerk configuratie}

% inter container communication

% ip tables

% Tls authentication

\subsection{registry configuraion}


\subsection{Permissies voor Docker config files}

Het configureren van de Docker daemon met de juiste instellingen voor zowel networking, registry gebruik en algemene configuratie heeft enkel nut als we ook de configuratie files waarin deze instellingen worden bewaard correct beveiligd zijn. 

% TODO Docker.Service file use?
% TODO DOcker-registry.service file use?
% TODO Docker.socket file use?
% TODO Docker environment file? whut 
% TODO Docker-network invironment file? 
% TODO docker-registry environment file?
% TODO Docker-storage environment file?
% TODO what is stored in /etc/docker folder?
% 


\section{Security of Docker images}

Als men het hebben over de beveiliging van Docker images zijn er twee grote categoriën die moeten onderscheiden worden. Een eerste categorie is wanneer we werken met Docker images die afkomstig zijn van de officiële Docker Hub Registry, een repository waar iedereen zijn images op kan posten en kan distrubueren. Daarnaast hebben we ook de mogelijkheid om onze eigen images te maken.

\subsection{Images van de Docker hub}

Wanneer we Docker images gebruiken van de officiële Docker Hub Registry is een handig hulpmiddel het commando ´docker search zoekterm´. Dit geeft ons een lijst van alle Docker images die voldoen aan deze zoekterm. Een voorbeeld hiervan vindt u hieronder.

%TODO fukcing kut tables

\begin{lstlisting}[language=bash]
$ docker search busybox
\end{lstlisting}
\begin{table}[!ht]
	\scriptsize
	\centering
	\begin{tabular}{lllll}
		NAME                 & DESCRIPTION                                   & STARS & OFFICIAL & AUTOMATED \\
		busybox              & Busybox base image.                           & 316            & [OK]              &                    \\
		progrium/busybox     &                                               & 50             &                   & [OK]               \\
		radial/busyboxplus   & Full-chain, Internet enabled, busybox made... & 8              &                   & [OK]               \\
		odise/busybox-python &                                               & 2              &                   & [OK]              
	\end{tabular}
\end{table}



Een van de dingen die hierbij opvallen is de kolom `OFFICIAL`. Een image die als "official" is aangeduid, is door het goedkeuringsproces van Docker zelf gegaan. Dit houdt onder andere in dat de images onderhouden worden, bij herhaaldelijk builden altijd hetzelfde resultaat bekomen wordt, consistent maar toch duidelijk zijn en goed beveiligd zijn. Meer hierover in het deel over eigen Docker images beveiligen. 

De officiële images zijn in het algemeen veiliger om te gebruiken dan niet officiële images. Dit wil niet zeggen dat onofficiële images niet veilig kunnen zijn, maar deze zijn niet gecontroleerd geweest door Docker zelf. Doordat we bij het downloaden van een Docker image niet direct kunnen zien wat deze zal doen, wordt het dus sterk afgeraden om Docker images van niet vertrouwelijke bronnen te gebruiken.

Van elke image die gevonden wordt met ´docker search´ en dus op de officiële repositories staat (ervan uitgaand dat er manueel geen andere repositories zijn toegevoegd) is online op de Docker hub webpagina een Dockerfile te vinden. Met deze Dockerfile kunnen we ook zelf kijken hoe de image precies is opgebouwd. Dit kan wel snel een tijdrovend proces worden aangezien Docker images kunnen gebaseerd zijn op andere images. Hierdoor kan voorkomen dat er meer dan een paar Dockerfiles moeten bekeken worden voordat men weet wat er precies allemaal inbegrepen zit in de onderzochte image.

Een ander belangrijk element dat met de officiële Docker images probeert opgelost te worden, is dat iedereen zijn eigen manier heeft om de Dockerfiles op te bouwen en software te installeren. Door een methodiek aan te bieden die gevolgd moet worden voordat de image kan opgenomen worden in de officiële Docker hub zorgen ze voor een gelijkaardige opbouw van Dockerfiles bij officiële Docker images.

%TODO not happy vv

De vergelijking met een virtuele machine zonder Docker kan gemaakt worden door de officiële Docker hub te vergelijken met een package archive, waarvan programma's kunnen geïnstalleerd worden in de meeste linux distributies. In beide gevallen kan zowel van de officiële repositories afgehaald worden als zelf niet officiële repositories toegevoegd. Het verschil ligt dan uiteraard in de aangeboden software, Docker images t.o.v. packages. 

\subsection{Zelf Docker Images maken}

Als we zelf Docker images willen maken zijn de eisen die gesteld worden voor officiële Docker images een goede referentie. Uiteraard zijn er hier elementen die voor eigen gebruik niet uiterst noodzakelijk zijn, maar deze vormen een goede basis om van te vertrekken.

Als eerste belangrijk punt moeten we rekening houden met welke packages we willen gebruiken in onze images. Hierbij zijn dezelfde risico's aan verbonden net zoals bij het gebruiken van een virtuele machine zonder Docker. Enkel packages installeren die gebruikt worden en het goed configureren (indien nodig) van packages is hier even belangrijk als bij een gewone virtuele machine. het is daarom aangewezen om bij het installeren van packages de fingerprint te controleren. 

Bij het maken van onze Docker images kunnen we ook bestanden kopiëren in onze container. Dit wordt best zo veel mogelijk vermeden maar indien toch vereist, is het best om de file specifiek te kopiëren en niet een hele map. Bij het copiëren van een volledige map kunnen bij herbuilden ongewenste files in de map bijgekomen zijn, waardoor er een onverwacht of ongewenst resultaat kan bekomen worden. 

Een belangrijk element om bij stil te staan als de gemaakte Docker image robuust wenst gemaakt te worden, is het definiëren van vaste package versies. Hierdoor blijven onze images bij het herbuilden van dezelfde versie altijd hetzelfde. Extra beveiliging kan hierdoor verkregen worden, door het gebruiken van specifieke versies van packages weten we ook altijd exact welke versies van deze packages er in onze image gebruikt worden. Een nadeel hieraan is dat bij het uitkomen van nieuwe updates en eventuele security updates en bugfixes we onze Docker images moeten aanpassen. In het algemeen wordt aangeraden om voorgedefinieerde versies te gebruiken, nieuwe versies van software worden best eerst uitvoerig getest vooraleer in gebruik genomen te worden. Door het definiëren van de versie die we willen gebruiken hebben we hierover volledige controle.

Het maken van onze eigen Docker images kunnen we dus volledig vergelijken met het configureren van een virtuele machine zonder Docker. Hierbij moet er ook gekeken worden naar welke bestanden we willen gebruiken op deze machine alsook welke packages er op geïnstalleerd moeten staan. Dezelfde veiligheidsvoorschriften die gebruikt worden bij het configureren van een virtuele machine zijn dus ook van toepassing op het maken van een eigen Docker image.



\section{Container configuration}

Als we Docker containers gebruiken, moeten de containers zelf ook goed geconfigureerd zijn. Een groot deel van de veiligheid van de containers zelf hangt af van hoe het ´docker run´ commando gebruikt wordt. Daarnaast zijn er ook nog een aantal punten waarvoor er moet worden oppast of die kunnen geoptimaliseerd worden op de container zelf.

\subsection{docker run commando configureren}

Elke container word opgestart met het commando ´docker run [OPTIONS] IMAGE [COMMAND] [ARG...]´. Met dit commando kunnen aan de hand van onze zelfgemaakt Docker image of een gedownloade Docker image een container aanmaken en direct opstarten. Doordat dit een van de belangrijkste commando's is die bij Docker hoort, is het uiteraard ook een van de krachtigste. Er moeten dus wel beter  enkele dingen in gedachten gehouden worden bij het uitvoeren van dit commando.

Bij het uitvoeren van het standaard ´docker run´ commando, namelijk ´docker run IMAGE´ zijn er veel dingen die standaard in orde zijn. Uiteraard zijn er dingen waarmee er normaal geen problemen mee mogen zijn maar uitzonderlijk kunnen hier toch problemen opduiken.

Eén van deze dingen is de standaard Linux Kernel Capabilities die de container heeft. De werking Linux Kernel Capabilities valt buiten de scope van deze Bachelorproef en wordt daarom ook hier niet uitgebreid behandeld. Maar sterk vereenvoudigd is dit de basis instructieset die een linux besturingssysteem bezit, deze instructies zijn als gewone gebruiker niet allemaal toegankelijk maar de root user kan elke instructie uitvoeren. Docker heeft uit voorzorg hiervoor deze lijst met Capabilities al sterk verminderd in vergelijking met de gewone Linux Kernel Capabilities. Dit zorgt ervoor dat de root user in een container veel minder kan dan een root user op een gewoone machine. Bij meeste use cases is het ook niet noodzakelijk om de containers volledige root rechten te verschaffen. Hiervoor kunnen we dus de Linux Kernel Capabilities per container gaan tweaken en enkel privileges toewijzen aan de containers die ze echt nodig hebben. Om deze beperking te omzeilen, kunnen Docker containers ook met de optie ´--privileged´ opstarten. Dit geeft de container alle Kernel Capabilities die het hostsysteem heeft en wordt dus best, buiten in enkele zeer specifieke use cases, vermeden.

Naast de mogelijkheid om de standaard Kernel Cabapilities aan te passen van de containers hebben we ook de mogelijkheid om het processor en geheugen gebruik te limiteren. Docker containers kunnen zonder verdere configuratie het complete geheugen gebruiken van de host, hierdoor kunnen andere containers zonder geheugen geraken en dit kan voor problemen zorgen. De mogelijkheid dat een container het complete geheugen van het hostsysteem in beslag neemt, is zeker iets waar rekening mee gehouden moet worden. Hiervoor bestaat de optie ´-m´. Hiermee kan de maximale hoeveelheid geheugen die een container mag gebruiken bepaald worden. Een optie om de minimale hoeveelheid geheugen voor containers met een kritieke rol vast te leggen ontbreekt daarbij heelaas. Hierdoor kunnen deze kritieke containers ook zonder geheugen geraken waardoor de werking van ons systeem in gedrang komt. Een mogelijkheid om dit te vermijden zou kunnen zijn, aan elke container een bepaalde hoeveelheid geheugen toewijzen waardoor als de som gemaakt wordt van alle geheugengebruik die som lager blijft dan de maximale hoeveelheid geheugen dat de host ter beschikking kan stellen. Maar dan moet elke keer als een nieuwe container teogevoegd wordt elke container opnieuw opstarten met een andere hoeveelheid toegewezen geheugen. Hierdoor wordt er verloren aan flexibiliteit die  net verkregen wordt door Docker te gebruiken, en dit is dus niet aan te raden. Daarnaast is er ook het gebruik van de Central Processing Unit (CPU). Elke container gebruikt standaard een gelijke hoeveelheid van de CPU. De optie die gebruikt wordt om het cpu gebruik te bepalen is de ´-c´ of de cpu-shares optie. Standaard heeft deze een waarde van 1024, als alle containers deze waarde hebben, krijgen ze allemaal gelijke prioriteit op de cpu. Door deze waarde te verhogen of verlagen kan er desgewenst een hogere of lagere prioriteit ingesteld voor de container met deze aangepaste optie. 

Het is maar op het moment dat dit standaard run commando aangepast wordt met opties dat er enkele dingen zijn waarvoor er best wordt opgelet. Wanneer een Docker run commando wordt uitgevoerd, zijn er twee dingen die in opties kunnen toevoegd worden waar men toch iets voorzichtiger mee moet zijn. Namelijk netwerk configuratie en mounts van host mappen in de container.

Netwerkconfiguratie van Docker is, als er niets aan is aangepast, tamelijk robuust. Maar indien aan dit commando aangepast wordt, is bestaat er de kans dat deze robuustheid te ondermijnd wordt. Eén van de elementen hierbij is poorten. Bij het verbinden naar buiten de machine, waarop een container opereert, moet zowel een poort op de host als een poort op de container worden opengesteld. Op deze manier worden deze poorten verbonden, waardoor het netwerkverkeer die op de gedefinieerde poort van de host toekomt, doorgezonden wordt naar de verbonden poort op de container. Een best practice hierbij is om altijd de volledige connectie te definiëren, dit wil zeggen dat zowel de de externe interface, de externe host poort als de interne container poort meegeven wordt in het run commando. Hierbij worden op de container best enkel poorten geopend die effectief gebruikt worden door deze container. Dit zorgt ervoor dat het aanvalsvlak via het netwerk op deze container verkleint. TCP/IP poorten onder 1024 op de host machine worden best ook vermeden om op te binden. Uiteraard zijn er services die hier op moeten binden, zoals een HTTP-proxy die op poort 80 moet luisteren om te kunnen functioneren, die ook veiliger als ze hierop gebonden zijn. Een goede maatstaf om te weten opdat een container op een geprivilegieerde poort gebonden moet worden, is kijken indien men de service die in de container draait, ook wanneer die op een virtuele machine draait, op een poort onder 1024 zou binden. Daarnaast hebben we net zoals bij Kernel Capabilities ook bij networking een manier om het intern Docker netwerk te omzeilen. Een container gebruikt standaard een bridged netwerk. Hierdoor wordt deze container in een aparte netwerk stack gestoken. Als we de optie ´--net=host´ meegeven overschrijven we dit en geven we de container toegang tot alle netwerkinterfaces van de host machine. Dit moet, tenzij in zeer specifieke toepassingen, niet gebruikt worden.


%TODO Mounting van directories
\subsection{Container configuratie}

Om extra beveiliging te voorzien voor de Docker containers is het mogelijk (indien ondersteund door de host besturingssysteem) om AppArmor en SeLinux te gebruiken. Met AppArmor is het mogelijk om voor elke container apart een AppArmor profiel te maken. SeLinux voor Docker moet aangezet worden bij het opstarten van de Docker Daemon. Dit kan simpel gedaan worden door de Daemon bij het starten de optie ´--selinux-enabled´ mee te geven. Hierna kunnen we SeLinux security opties meegeven aan containers. Zowel AppArmor en Selinux zorgen ervoor dat we elke container individueel kunnen configureren op vlak van security. Het individueel configureren van SeLinux en AppArmor voor containers kan vergeleken worden met het configureren voor processen op een niet-Docker systeem. Dit is buiten de scope van deze Bachelorproef en hier wordt daarom niet verder op ingegaan.

Daarnaast is een element dat we moeten bekijken wat we nu precies in een container willen draaien qua processen. Het idee achter Docker is om één enkele applicatie per container op te delen. Docker luistert standaard enkel op één hoofdproces. Dus indien we meerdere processen willen uitvoeren op één container, die niet gemaakt zijn om vanuit één hoofdproces op te starten en informatie terug door te geven via dit hoofd proces, moeten we procesbeheer gaan bijvoegen in de container. Dit zorgt er onder andere voor dat een container, wat algemeen gezien een simpel iets is, weer zeer complex wordt. Anderzijds zorgt dit ervoor dat je de processen binnen de container niet meer op een optimale manier kunt monitoren met Docker. Een voorbeeld hiervan zou zijn indien zowel de applicatie als de database op eenzelfde container staat. Inden deze configuratie zou gebruikt worden, zal er al snel opgemerkt worden dat het heel wat makkelijker zou zijn om gewoon de database uit de applicatie container te halen en in zijn eigen container op te starten. Dit geeft ons naast een robuuster systeem (als de applicatie crasht zou het kunnen dat dit de database mee doet crashen indien het op dezelfde container draait) ook een meer schaalbaar systeem door gewoon meer applicatiecontainers te linken met dezelfde database container. Een ander voorbeeld van een nutteloos element om in een container te draaien is ssh, containers hebben in de meeste gevallen geen ssh nodig. Alle containers zijn namelijk al toegankelijk via het hostsysteem met het commando ´docker exec CONTAINER´ dus is daar de meest aangewezen plaats om SSH te installeren. Met een installatie van ssh op de hostmachine, kan er met een ssh connectie naar de host makkelijk ingelogd worden op elke container.

\chapter{Managebility}

Wanneer de managebility van Docker containers onderzocht wordt, zijn er drie opties, in de eerste plaats kan alles manueel gemanaged worden. Daarnaast kan  ook alles automatisch laten gemanaged worden door een proces manager. Ofwel kunnen er tools gebruikt worden om onze containers te beheren. Als men bezig is met het beheren van Docker containers moet er ook rekening gehouden worden met het beheren van onze docker images. 

\section{Manueel beheren}

Een optie waarmee meestal begonnen wordt is het manueel managen van de Docker containers. Bij deze optie zullen de docker containers beheren enkel met de commando's die docker ons aanbied. Elke container invividueel moeten aanroepen in elk commando lijkt mogelijk in het begin. Maar na het manueel uitvoeren van deze commando's wordt het al snel duidelijk datmet behulp van een kort script of een makefile het beheren heel wat makkelijker kan. Op deze manier kunnen we zowel onze containers managen als onze Docker images builden met veel kortere commando's. Containers managen met de volledige docker commando's of scripts/makefiles is doenbaar indien we enkele containers moeten managen zoals in een ontwikkel omgeving of een stabiele kleine omgeving. Het manueel managen van onze containers verwacht ook meer stabiliteit van onze containers. Indien er niemand de containers aan het monitoren zou zijn wanneer een container crasht, moet er gewacht worden tot er iemand deze manueel terug opstart. Hierdoor kan het dus gebeuren dat een container relatief langer onbeschikbaar is. In een development omgeving zijn de vereisten voor uptime veel lager. Hier is het manueel managen van onze containers en images wel een optie. Bij het manueel managen wordt er ook gebruik gemaakt van scripts of makefiles. Een voorbeeld van beide vind u hieronder.

\begin{lstlisting}[language=bash, style=configstyle]
CONTAINER ?= redis
IMAGE_NAME ?= redis
IMAGE_VERSION ?= latest
HOSTNAME ?= redis-host


# full image name
IMAGE = $(IMAGE_NAME):$(IMAGE_VERSION)
# full ssh connect vars
CONNECT = $(REMOTE_USER_NAME)@$(IP)

build:
	docker build -t $(IMAGE) .

run:
	docker run -d --name $(CONTAINER) -h $(HOSTNAME) $(IMAGE)

logs:
	docker logs $(CONTAINER)

kill_container:
	docker kill $(CONTAINER)

remove_container:
	docker rm $(CONTAINER)

remove_image:
	docker rmi $(IMAGE)

update: kill_container remove_container remove_image build run
\end{lstlisting}

Met deze makefile hebben wordt er toegang verkregen tot de verschillende commando's die normaal volledig met de hand zouden moeten uitgeschreven worden, maar nu kunnen deze aangeroepen worden met 'make COMMANDO'. Als we dit bekijken met de voorgaande makefile zien we dat om onze redis container uit te voeren enkel 'make run' moest ingeven worden in het terminal venster terwijl anders 'docker run --name redis redis' moest worden gebruikt. Bij dit voorbeeld is het commando dat zonder makefile zouden moeten uitgevoerd worden nog beperkt maar als er meer opties meegeven moeten worden met het commando kan het interessanter zijn  om een kort make commando uit te voeren dan elke keer dat dezelfde container  opgestart moet worden opnieuw het volledige commando uit te typen. Ook wordt consistentie behouden bij het uitvoeren van de commando's via Makefiles en kunnen eventuele menselijken fouten zoals typfouten makkelijker geëlimineerd worden. Het ander voordeel dat hier ook uit voorkomt is dat make commando's ook kunnen aanroepen worden in andere make commando's, hierdoor kan het commando 'make update' gemaakt worden. Deze verwijdert onze oude bestaande container en image. Daarna wordt een nieuwe image gemaakt en ingeladen in een nieuwe container. Bovenaan de makefile vinden we ook variabelen die gebruikt kunnen worden in onze make commando's. Met behulp van deze variabelen kunnen we een makefile snel aanpassen voor het gebruik met een ander commando.

%TODO insert script here

Bij het gebruik van scripts wordt ongeveer dezelfde functionaliteit verkregen als bij makefiles, enkel is de syntaxis voor het uitvoeren anders en de manier waarop het geschreven is. Hierboven vinden we een script die dezelfde functionaliteit zou hebben als het 'make update' commando.



\section{Managen met process manager}

Een andere optie die er is, zijn de containers managen met een process-manager. Hierbij kunnen de containers bekeken worden alsof het processen zijn. Als de  containers gemanaged worden met een process-manager, moeten configuratie bestanden handmatig geschreven worden voor de gekozen process-manager. Die gekozen manager zal dan deze bestanden inlezen en vervolgens de met de parameters vanuit de configuratie bestanden onze containers starten en managen zoals het een ander proces zou beheren. Een voorbeeld hiervan is het gebruik van Systemd Unit files. Dit is hieronder uitgewerkt voor de Redis Container uit hoofdstuk drie.

%TODO systemd unit files uitleggen

\begin{lstlisting}[language=bash, style=configstyle]
[Unit]
Description=Run a redis container
Author=Toon Lamberigts & Tomas Vercautter
Requires=docker.service

[Service]
ExecStartPre=-/usr/bin/docker rm redis-container
ExecStart=/usr/bin/docker run --rm --name redis-container -h redis-host redis
ExecStop=/usr/bin/docker stop -t 2 redis
\end{lstlisting}

Deze unit file zal bij het proberen starten van de bijbehorende service ingeladen worden. Het 'ExecStartPre' gedeelte wordt uitgevoerd voor de service effectief zal starten. Deze zal de container verwijderen indien deze bestaat, om zeker te zijn dat er geen container genaamd 'redis-container' bestaat. Dit zou een fout genereren bij het opstarten waardoor de service zou falen. Dus uit voorzorg proberen we deze container eerst te verwijderen voor het opstarten. Door het meegeven van een '-' voor het commando, zal het opstarten van deze service niet falen indien er geen container genaamd 'redis-container' bestaat (en het proberen verwijderen ervan een fout teruggeeft). Het onderdeel 'ExecStart' is het effectieve commando dat de service zal uitvoeren en zich aan zal binden. In dit geval is dat onze redis container. We geven bij het opstarten via unit files geen flag '-d', detached mee omdat we willen dat de container zich bind aan de service waardoor we de container kunnen monitoren via de aangemaakte service. Hiervoor geven we ook de '--rm' tag mee, deze zorgt ervoor dat bij het stoppen van de container hij ook direct verwijdert wordt. Dit gebeurt jammer genoeg enkel bij een correcte terminatie van de container. Dus indien de docker service crasht zal deze niet verwijderd worden maar wel gestopt.

\section{Managen met tools}

Een derde optie die bestaat is het managen van de containers met tools. In dit onderdeel wordt er niet uitgebreid beschrijven hoe elke tool werkt en hoe die moet gebruikt worden, dit moet geval per geval bekeken worden als dit een meerwaarde betekend voor het project met docker. In dit hoofdstuk wordt er geprobeert een voorbeeld te geven hoe tools zouden kunnen gebruiken worden in een docker omgeving. De tool waarover wat meer uitleg zal gegeven worden is Vagrant. Deze tool is enkel een voorbeeld en is zeker niet de enige tool. Vagrant is een tool ontwikkeld voor het gebruik in ontwikkelomgevingen en wordt door Hashicorp ontwikkeld. Daarnaast hebben ze ook Nomad, een nieuwe tool die in moet staan voor de deployment van containers. Andere bedrijven hebben andere tools en zoals hierboven vermeld moet voor elke use case gekeken worden hoe tools eventueel kunnen geïntegreerd worden in het ontwikkelproces met docker. Het voorbeeld van Vagrant dient om een idee de vormen van hoe tools gebruikt kunnen worden. Docker bied zelf ook een tool aan voor het managen van docker containers namelijk Docker Compose. Een voorbeeld van een configuratiebestand voor Vagrant en Docker Compose worden hieronder weergegeven. Beide zetten een docker container die redis noemt, op van een image genaamd redis.

\begin{lstlisting}[language=bash, style=configstyle]
Vagrant.configure("2") do |config|
	config.vm.provision "docker" do |d|
		d.image = "redis"
		d.name = "redis"
	end
end
\end{lstlisting}
Dit is een voorbeeld van een Vagrant bestand voor docker. Deze kan opgestart worden met het commando 'vagrant up'


\begin{lstlisting}[language=bash, style=configstyle]
#redis image
redis:
	image: redis
\end{lstlisting}
Dit is een voorbeeld van een Docker-compose bestand voor docker. Deze kan opgestart worden met het commando 'docker-compose up'


\chapter{Logging}

Logging is een zeer belangrijk onderdeel van docker containers. Omdat we onze docker containers veel meer gaan isoleren en onze containers veel vluchtiger zijn dan gewone services op een virtuele host wordt loggen van
\chapter{Conclusie}
\label{ch:conclusie}

% TODO: Trek een duidelijke conclusie, in de vorm van een antwoord op de
% onderzoeksvra(a)g(en). Reflecteer kritisch over het resultaat. Zijn er
% zaken die nog niet duidelijk zijn? Heeft het ondezoek geleid tot nieuwe
% vragen die uitnodigen tot verder onderzoek?

Er zijn sinds Docker uitgekomen is heel wat kritieken geweest over de werking van Docker. Zowel gerechtvaardigde als ongerechtvaardigde. Docker biedt ons een tool die flexibeler en complexer is dan het gebruik van een gewone virtuele machine. Er zijn hier wat foute opvattingen over Docker geweest in verband met vooral security maar ook op vlak van managebility. Er werden verwachtingen gesteld dat er met Docker niet meer extra moest beveiligd worden, dat op welke manier we onze containers willen gebruiken, ze automatisch veilig en managable zijn. Op het moment dat wat we normaal op een virtuele machine zouden draaien, gaan containerizen moeten we de beveiliging gewoon op een andere plaats voorzien. Als we gebruik maken van virtuele machines moet er nog altijd gebruik gemaakt worden van beveiliging van het netwerk van de virtuele machines. Daarnaast werd ook elke virtuele machine op een andere manier beveiligd. Deze best practices die we vroeger toepasten op de virtuele machines, mogen we nu niet vergeten bij het gebruik van Docker. Elke Docker container moet nog altijd beveiligd worden naargelang wat we willen uitvoeren op de container. Als we Docker gebruiken maken we gebruik van een flexibele, complexe en krachtige tool. zoals Voltaire zei, "With great power comes great responibility". Docker is een relatief nieuwe tool waarvoor een actieve community is die problemen die opkomen zo snel mogelijk oplossen. Als we onze containers op een virtuele machine uitvoeren en zowel de containers als virtuele machine correct beveiligen en up to date houden, is er zeker een veilige plaats voor Docker in een serveromgeving.

Door de extra flexibiliteit die Docker aanbied, wordt het moelijker om onze containers te managen. Er moet kunnen afgestapt worden van het minutieus controleren van containers. De containers moeten zo gemaakt worden dat er zonder veel menselijke tussenkomst langdurig gebruik van gemaakt kan worden. Er kunnen tools voor het managen van Docker containers zowel voor productie als development gebruikt worden. Bij het gebruik van Docker containers wordt logging en monitoring heel wat belangrijker. Wanneer we onze containers meer willen "loslaten" moet er een systeem bestaan die ervoor instaat ons te waarschuwen als er iets misloopt met onze containers.

Docker bied ons een extra laag van virtualisatie aan. Hierdoor mogen er niet dezelfde eisen stellen aan Docker als dat we doen bij virtuele machines. Beide hebben hun plaats op de markt, en hebben de bedoeling om met elkaar samen te werken en niet elkaar vervangen. Hierbij moeten we rekening houden als we wat vroeger op verschillende virtuele machines zou uitgevoerd worden nu laten uitvoeren op één virtuele machine met Docker containers. We een deel van de controle over deze machines verliezen, maar veel meer flexibiliteit verkrijgen. Daarom mag er niet vergeten worden om de veiligheids elementen die vroeger bestonden om de aparte virtuele machines te beveiligen ook worden meegebracht naar het nieuwe systeem met Docker containers.

\nocite{*}

\bibliographystyle{apa}
\bibliography{bachproef}

%%---------- Back matter -------------------------------------------------

\listoffigures
\listoftables

\end{document}

%moeilijke woorden: linux kernel capabilities
