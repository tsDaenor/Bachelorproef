\def\signed #1{{\leavevmode\unskip\nobreak\hfil\penalty50\hskip2em
		\hbox{}\nobreak\hfil(#1)%
		\parfillskip=0pt \finalhyphendemerits=0 \endgraf}}

\newsavebox\mybox
\newenvironment{aquote}[1]
{\savebox\mybox{#1}\begin{quotation}}
	{\signed{\usebox\mybox}\end{quotation}}

\section{Logging en monitoring}

We kunnen onze Docker containers zo goed beveiligen en managen als we willen. Het is bijna nooit de vraag "zal er iets verkeerd gaan?", maar eerder de vraag wanneer iets verkeerd zal gaan. Om te kunnen weten wanneer er iets verkeerd gaat en wat er precies verkeerd gaat, is het monitoren van onze containers en loggen van wat er gebeurd zeer belangrijk. Daarom wordt er hier kort behandeld wat een mogelijke manier is om containers te monitoren en hoe deze containers kunnen worden gelogd. De tools die hier zullen uitgelegd zijn opnieuw zeker niet de enige die hiervoor gebruikt kunnen worden. De focus ligt hier eerder op wat er gemonitord en gelogd moet worden en hoe dit persistent kan gebeuren.
\begin{aquote}{Patrick Debois} De mogelijkheid bestaat om zelf een volledig systeem op te zetten om het systeem te loggen en monitoren. Maar het is heel wat belangerijker om te weten wat er moet gelogd en gemonitord worden dan de manier waarop we dit doen. Het is dan ook individueel uit te maken als er gekozen wordt om logging en monitoring zelf te voorzien of te kiezen voor een bijna kant en klare oplossing.
\end{aquote}

Bij het loggen en monitoren van containers is er niet zo veel veranderd in vergelijking met het monitoren en loggen van een virtuele machine. er bestaad nog altijd zowel de optie om alles zelf te configureren of een tool installeren die het voor ons doet. Hierbij moet er enkel rekening gehouden worden met het feit dat er nu, op de machine waarop de containers staan, evenveel logs worden geproduceerd als bij verschillende virtuele machine samen wanneer er zonder docker containers wordt gewerkt. Bij monitoren van containers is het belangrijk dat zowel het hostsysteem wordt gemonitord, als de services of software in de containers. Waarbij we bij het monitoren van het hostsysteem nog altijd dezelfde elementen monitoren als bij een gewone virtuele machine, maar nu ook docker specifieke elementen moeten bijhouden zoals onder andere verkeer tussen containers en welke containers zich op dat hostsysteem bevinden.


%En welke dingen er moeten gemonitord worden, zijn de elementen die in een bepaalde opstelling een kritieke rol spelen. En het is belangrijk om te weten welke dat zijn.

%Geen idee lol. Maar denk dat het er vooral op neerkomt dat het nutteloos is om een volledig fancy logging/monitoring systeem op te zetten als ge niet weet hoe die data te analyseren