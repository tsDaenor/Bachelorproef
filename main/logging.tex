\chapter{Logging en monitoring}

We kunnen onze Docker containers zo goed beveiligen en managen als we willen, het is bijna nooit de vraag zal er iets verkeerd gaan? Maar eerder de vraag wanneer iets verkeerd zal gaan. Om te kunnen weten wanneer er iets verkeerd gaat en wat er precies verkeerd gaat is het monitoren van onze containers en loggen van wat er gebeurd zeer belangerijk. Daarom heb ik besloten van kort om kort uit te leggen wat een mogelijke manier is om onze containers te monitoren en hoe we onze containers kunnen loggen. De tools die ik hier zal uitleggen zijn zeker niet de enige die hiervoor gebruikt kunnen worden. De focus ligt hier eerder op wat te monitoren en loggen en hoe we dit persistent kunnen doen dan op hoe we dit doen. We zouden kunnen zelf alles opzetten om te monitoren en loggen maar waarom zoveel tijd steken in hoe alles te loggen terwijl we kunnen bezig zijn met het belangerijker aspect namelijk wat te monitoren en loggen \quote{Patrick Debois}. 

Bij het loggen en monitoren van onze containers is er niet zo veel verandert in vergelijking met het monitoren en loggen van een virtuele machine. We hebben nog altijd zowel de optie om alles zelf te configureren of een tool installeren die het voor ons doet. Hierbij moet er enkel rekening gehouden worden met het feit dat er nu op de machine waar de containers op staan evenveel logs worden geproduceerd als bij een aanpak zonder containers op verschillende virtuele machines samen worden geproduceerd. Bij monitoren van containers is het belangerijk dat zowel het hostsysteem wordt gemonitord)als de services of software in de containers. Waarbij we bij het monitoren van het hostsysteem nog altijd de zelfde elementen monitoren als bij een gewone virtuele machine maar nu ook docker specifieke elementen moeten bijhouden, zoals onder andere traffic tussen containers en welke containers zich bevinden op dat hostsysteem.