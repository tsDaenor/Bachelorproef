\chapter{Security}

Op vlak van security is er gestart van de CIS Docker 1.6 Benchmark geschreven door \cite{PravinGoyal2015}. Hierin is in samenwerking met de "Center for Internet Security", onderzoek gedaan naar de beveiliging van Docker containers. Dit onderzoek is onderverdeeld in zes grote onderdelen namelijk Host configuration, Docker daemon configuration, Docker daemon configuration files, Container images and build file, Container runtime en Docker security operations. Hierbij wordt er gekeken welke van deze elementen standaard in Docker worden uitgevoerd. In dit hoofdstuk zullen dezelfde zes onderdelen behandeld en vergeleken worden. Hierbij zal er daarnaast gekeken worden hoe we deze beveiliging aanpakken wanneer onze services in een Docker container uitgevoerd worden, tegenover wanneer deze native in onze virtuele machine draaien.


Hierbij zijn ook telkens deze methodes voor Docker containers uitgetest om te kijken hoe makkelijk deze kunnen geïmplementeerd worden in een nieuwe omgevinge of bestaande omgeving. Een grote hulp hiervoor was de github repository "docker-bench-security" van Docker een grote hulp. Dit is een script die alle elementen die worden aangehaald in CIS Docker 1.6 Benchmark automatisch getest en feedback over gegeven.

\section{Host Configuration}

Een van de belangrijkste onderdelen in het beveiligen van de Docker stack, zijnde de combinatie van de docker host machine en de docker containers, is het beveiligen van de host machine waarop de Docker containers zich zullen bevinden. Dit heeft twe grote onderdelen, namelijk het up-to-date houden van het systeem en het juist configureren van het systeem.

\subsection{Up-to-date houden van het systeem}

Om de veiligheid van een systeem te kunnen garanderen moet er ten allen tijde gezorgd worden dat het systeem up-to-date blijft. Dit zorgt ervoor dat veiligheidsgebreken die gekend en gepatcht zijn ook op het systeem beveiligd zijn. Doordat we Docker gebruiken komt hier nog een extra laag bij. Bij het Updaten van een machine waarop de services rechtstreeks draaien moeten enkel het host systeem, de kernel van het host systeem en de services up to date blijven. Doordat Docker een kernel deelt met het hostsysteem wordt het up to date houden van de kernel belangrijker. Naast het onderhouden van de voorvernoemde elementen moet er nu, doordat we Docker gebruiken, nog met twee extra dingen rekening gehouden worden. Enerzijds moet Docker zelf nu ook bijgewerkt worden, maar een onderdeel dat soms vergeten wordt, is dat elke container nu ook zijn eigen besturingssysteem gebruikt. Hierover wordt er verder bij het beveiligen van Container images meer uitleg over gegeven.

% TODO dev tools? and explanaition about stable releases

\subsection{Configureren van het systeem}

Bij het configureren van het hostysteem wordt net zoals bij het up to date houden van het systeem dezelfde stappen van een native applicatie op een hostysteem configureren gevolgd. Maar door de extra laag komen daar nog extra stappen bij. Bij een systeem waar Docker op draait, wordt er aangeraden om een aparte partitie te creëren voor de containers. Dit is een simpele operatie waardoor alle Docker gerelateerde bestanden zich niet meer tussen de bestanden van het hostsysteem bevinden.

% TODO https://docs.docker.com/articles/security/#docker-daemon-attack-surface

Daarnaast moet ook onder controle gehouden worden wie toegang krijgt tot de Docker daemon. De Docker daemon heeft 'root' access, gebruikers die worden toegevoegd aan de 'docker' usergroup en geen 'root' privileges hebben kunnen hierdoor 'root' access verkrijgen tot het hostsysteem. Men kan mappen sharen tussen het host systeem en een guest container en zo toegang verkrijgen tot de gemounte mappen. Doordat containers standaard altijd runnen als 'root' heeft de container, als de '/' map gemount is op deze container, ongelimiteerde toegang tot het volledige host systeem. Hierdoor kunnen zonder restricties aanpassingen aan het hostysteem gedaan worden. Dit betekent dat een gebruiker die toegang heeft tot de Docker daemon verhoogde rechten kan verkrijgen door gewoon een container op te starten.

Zoals bij een virtuele machine waarop de services rechtstreeks worden uitgevoerd, wordt ook bij een systeem waar Docker containers op draaien aangeraden overbodige services uit te schakelen of te verwijderen. Moest een gebruiker door een van deze services toegang krijgen tot het hostsysteem, zou hij makkelijker 'root' rechten kunnen verkrijgen volgens de manier beschreven in de vorige paragraaf.


\section{Docker daemon configuration}

Bij het beveiligen van de docker stack is ook de configuratie van de docker daemon een belangrijk element. Dit heeft twee grote onderdelen. Enerzijds de configuratie van de docker daemon zelf, dit houd in dat we de daemon zelf zo gaan configureren dat deze zo veilig en robuust mogelijk wordt. Aan de andere kant moeten we ook zorgen dat de configuratie files zelf de juiste file permissies en owners hebben. Hierbij is de vergelijking met een virtuele machine, waar de services rechtstreeks op draaien, moeilijk aangezien er hier geen docker daemon op te vinden is. Omdat het configureren van de Docker daemon dingen verandert doe voor elke container toegepast worden kunnen we het wel vergelijken met hoe de inviduele services globaal geconfigureerd worden.

% TODO find information about docker binding on IP/Port or UNIX socket
% https://docs.docker.com/articles/basics/#bind-docker-to-another-hostport-or-a-unix-socket

\subsection{Globale configuratie}

% TODO find info about Docker switch from lxs to libcontainer

% lxc
% logging level
% storage driver

\subsection{netwerk configuratie}

% inter container communication

% ip tables

% Tls authentication

\subsection{registry configuraion}


\subsection{Permissies voor Docker config files}

Het configureren van de Docker daemon met de juiste instellingen voor zowel networking, registry gebruik en algemene configuratie heeft enkel nut als we ook de configuratie files waarin deze instellingen worden bewaard correct beveiligd zijn. 

% TODO Docker.Service file use?
% TODO DOcker-registry.service file use?
% TODO Docker.socket file use?
% TODO Docker environment file? whut 
% TODO Docker-network invironment file? 
% TODO docker-registry environment file?
% TODO Docker-storage environment file?
% TODO what is stored in /etc/docker folder?
% 


\section{Security of Docker images}

Als men het hebben over de beveiliging van Docker images zijn er twee grote categoriën die moeten onderscheiden worden. Een eerste categorie is wanneer we werken met Docker images die afkomstig zijn van de officiële Docker Hub Registry, een repository waar iedereen zijn images op kan posten en kan distrubueren. Daarnaast hebben we ook de mogelijkheid om onze eigen images te maken.

\subsection{Images van de Docker hub}

Wanneer we Docker images gebruiken van de officiële Docker Hub Registry is een handig hulpmiddel het commando ´docker search zoekterm´. Dit geeft ons een lijst van alle Docker images die voldoen aan deze zoekterm. Een voorbeeld hiervan vindt u hieronder.

%TODO fukcing kut tables

\begin{lstlisting}[language=bash]
$ docker search busybox
\end{lstlisting}
\begin{table}[!ht]
	\scriptsize
	\centering
	\begin{tabular}{lllll}
		NAME                 & DESCRIPTION                                   & STARS & OFFICIAL & AUTOMATED \\
		busybox              & Busybox base image.                           & 316            & [OK]              &                    \\
		progrium/busybox     &                                               & 50             &                   & [OK]               \\
		radial/busyboxplus   & Full-chain, Internet enabled, busybox made... & 8              &                   & [OK]               \\
		odise/busybox-python &                                               & 2              &                   & [OK]              
	\end{tabular}
\end{table}



Een van de dingen die hierbij opvallen is de kolom `OFFICIAL`. Een image die als "official" is aangeduid, is door het goedkeuringsproces van Docker zelf gegaan. Dit houdt onder andere in dat de images onderhouden worden, bij herhaaldelijk builden altijd hetzelfde resultaat bekomen wordt, consistent maar toch duidelijk zijn en goed beveiligd zijn. Meer hierover in het deel over eigen Docker images beveiligen. 

De officiële images zijn in het algemeen veiliger om te gebruiken dan niet officiële images. Dit wil niet zeggen dat onofficiële images niet veilig kunnen zijn, maar deze zijn niet gecontroleerd geweest door Docker zelf. Doordat we bij het downloaden van een Docker image niet direct kunnen zien wat deze zal doen, wordt het dus sterk afgeraden om Docker images van niet vertrouwelijke bronnen te gebruiken.

Van elke image die gevonden wordt met ´docker search´ en dus op de officiële repositories staat (ervan uitgaand dat er manueel geen andere repositories zijn toegevoegd) is online op de Docker hub webpagina een Dockerfile te vinden. Met deze Dockerfile kunnen we ook zelf kijken hoe de image precies is opgebouwd. Dit kan wel snel een tijdrovend proces worden aangezien Docker images kunnen gebaseerd zijn op andere images. Hierdoor kan voorkomen dat er meer dan een paar Dockerfiles moeten bekeken worden voordat men weet wat er precies allemaal inbegrepen zit in de onderzochte image.

Een ander belangrijk element dat met de officiële Docker images probeert opgelost te worden, is dat iedereen zijn eigen manier heeft om de Dockerfiles op te bouwen en software te installeren. Door een methodiek aan te bieden die gevolgd moet worden voordat de image kan opgenomen worden in de officiële Docker hub zorgen ze voor een gelijkaardige opbouw van Dockerfiles bij officiële Docker images.

%TODO not happy vv

De vergelijking met een virtuele machine zonder Docker kan gemaakt worden door de officiële Docker hub te vergelijken met een package archive, waarvan programma's kunnen geïnstalleerd worden in de meeste linux distributies. In beide gevallen kan zowel van de officiële repositories afgehaald worden als zelf niet officiële repositories toegevoegd. Het verschil ligt dan uiteraard in de aangeboden software, Docker images t.o.v. packages. 

\subsection{Zelf Docker Images maken}

Als we zelf Docker images willen maken zijn de eisen die gesteld worden voor officiële Docker images een goede referentie. Uiteraard zijn er hier elementen die voor eigen gebruik niet uiterst noodzakelijk zijn, maar deze vormen een goede basis om van te vertrekken.

Als eerste belangrijk punt moeten we rekening houden met welke packages we willen gebruiken in onze images. Hierbij zijn dezelfde risico's aan verbonden net zoals bij het gebruiken van een virtuele machine zonder Docker. Enkel packages installeren die gebruikt worden en het goed configureren (indien nodig) van packages is hier even belangrijk als bij een gewone virtuele machine. het is daarom aangewezen om bij het installeren van packages de fingerprint te controleren. 

Bij het maken van onze Docker images kunnen we ook bestanden kopiëren in onze container. Dit wordt best zo veel mogelijk vermeden maar indien toch vereist, is het best om de file specifiek te kopiëren en niet een hele map. Bij het copiëren van een volledige map kunnen bij herbuilden ongewenste files in de map bijgekomen zijn, waardoor er een onverwacht of ongewenst resultaat kan bekomen worden. 

Een belangrijk element om bij stil te staan als de gemaakte Docker image robuust wenst gemaakt te worden, is het definiëren van vaste package versies. Hierdoor blijven onze images bij het herbuilden van dezelfde versie altijd hetzelfde. Extra beveiliging kan hierdoor verkregen worden, door het gebruiken van specifieke versies van packages weten we ook altijd exact welke versies van deze packages er in onze image gebruikt worden. Een nadeel hieraan is dat bij het uitkomen van nieuwe updates en eventuele security updates en bugfixes we onze Docker images moeten aanpassen. In het algemeen wordt aangeraden om voorgedefinieerde versies te gebruiken, nieuwe versies van software worden best eerst uitvoerig getest vooraleer in gebruik genomen te worden. Door het definiëren van de versie die we willen gebruiken hebben we hierover volledige controle.

Het maken van onze eigen Docker images kunnen we dus volledig vergelijken met het configureren van een virtuele machine zonder Docker. Hierbij moet er ook gekeken worden naar welke bestanden we willen gebruiken op deze machine alsook welke packages er op geïnstalleerd moeten staan. Dezelfde veiligheidsvoorschriften die gebruikt worden bij het configureren van een virtuele machine zijn dus ook van toepassing op het maken van een eigen Docker image.



\section{Container configuration}

Als we Docker containers gebruiken, moeten de containers zelf ook goed geconfigureerd zijn. Een groot deel van de veiligheid van de containers zelf hangt af van hoe het ´docker run´ commando gebruikt wordt. Daarnaast zijn er ook nog een aantal punten waarvoor er moet worden oppast of die kunnen geoptimaliseerd worden op de container zelf.

\subsection{docker run commando configureren}

Elke container word opgestart met het commando ´docker run [OPTIONS] IMAGE [COMMAND] [ARG...]´. Met dit commando kunnen aan de hand van onze zelfgemaakt Docker image of een gedownloade Docker image een container aanmaken en direct opstarten. Doordat dit een van de belangrijkste commando's is die bij Docker hoort, is het uiteraard ook een van de krachtigste. Er moeten dus wel beter  enkele dingen in gedachten gehouden worden bij het uitvoeren van dit commando.

Bij het uitvoeren van het standaard ´docker run´ commando, namelijk ´docker run IMAGE´ zijn er veel dingen die standaard in orde zijn. Uiteraard zijn er dingen waarmee er normaal geen problemen mee mogen zijn maar uitzonderlijk kunnen hier toch problemen opduiken.

Eén van deze dingen is de standaard Linux Kernel Capabilities die de container heeft. De werking Linux Kernel Capabilities valt buiten de scope van deze Bachelorproef en wordt daarom ook hier niet uitgebreid behandeld. Maar sterk vereenvoudigd is dit de basis instructieset die een linux besturingssysteem bezit, deze instructies zijn als gewone gebruiker niet allemaal toegankelijk maar de root user kan elke instructie uitvoeren. Docker heeft uit voorzorg hiervoor deze lijst met Capabilities al sterk verminderd in vergelijking met de gewone Linux Kernel Capabilities. Dit zorgt ervoor dat de root user in een container veel minder kan dan een root user op een gewoone machine. Bij meeste use cases is het ook niet noodzakelijk om de containers volledige root rechten te verschaffen. Hiervoor kunnen we dus de Linux Kernel Capabilities per container gaan tweaken en enkel privileges toewijzen aan de containers die ze echt nodig hebben. Om deze beperking te omzeilen, kunnen Docker containers ook met de optie ´--privileged´ opstarten. Dit geeft de container alle Kernel Capabilities die het hostsysteem heeft en wordt dus best, buiten in enkele zeer specifieke use cases, vermeden.

Naast de mogelijkheid om de standaard Kernel Cabapilities aan te passen van de containers hebben we ook de mogelijkheid om het processor en geheugen gebruik te limiteren. Docker containers kunnen zonder verdere configuratie het complete geheugen gebruiken van de host, hierdoor kunnen andere containers zonder geheugen geraken en dit kan voor problemen zorgen. De mogelijkheid dat een container het complete geheugen van het hostsysteem in beslag neemt, is zeker iets waar rekening mee gehouden moet worden. Hiervoor bestaat de optie ´-m´. Hiermee kan de maximale hoeveelheid geheugen die een container mag gebruiken bepaald worden. Een optie om de minimale hoeveelheid geheugen voor containers met een kritieke rol vast te leggen ontbreekt daarbij heelaas. Hierdoor kunnen deze kritieke containers ook zonder geheugen geraken waardoor de werking van ons systeem in gedrang komt. Een mogelijkheid om dit te vermijden zou kunnen zijn, aan elke container een bepaalde hoeveelheid geheugen toewijzen waardoor als de som gemaakt wordt van alle geheugengebruik die som lager blijft dan de maximale hoeveelheid geheugen dat de host ter beschikking kan stellen. Maar dan moet elke keer als een nieuwe container teogevoegd wordt elke container opnieuw opstarten met een andere hoeveelheid toegewezen geheugen. Hierdoor wordt er verloren aan flexibiliteit die  net verkregen wordt door Docker te gebruiken, en dit is dus niet aan te raden. Daarnaast is er ook het gebruik van de Central Processing Unit (CPU). Elke container gebruikt standaard een gelijke hoeveelheid van de CPU. De optie die gebruikt wordt om het cpu gebruik te bepalen is de ´-c´ of de cpu-shares optie. Standaard heeft deze een waarde van 1024, als alle containers deze waarde hebben, krijgen ze allemaal gelijke prioriteit op de cpu. Door deze waarde te verhogen of verlagen kan er desgewenst een hogere of lagere prioriteit ingesteld voor de container met deze aangepaste optie. 

Het is maar op het moment dat dit standaard run commando aangepast wordt met opties dat er enkele dingen zijn waarvoor er best wordt opgelet. Wanneer een Docker run commando wordt uitgevoerd, zijn er twee dingen die in opties kunnen toevoegd worden waar men toch iets voorzichtiger mee moet zijn. Namelijk netwerk configuratie en mounts van host mappen in de container.

Netwerkconfiguratie van Docker is, als er niets aan is aangepast, tamelijk robuust. Maar indien aan dit commando aangepast wordt, is bestaat er de kans dat deze robuustheid te ondermijnd wordt. Eén van de elementen hierbij is poorten. Bij het verbinden naar buiten de machine, waarop een container opereert, moet zowel een poort op de host als een poort op de container worden opengesteld. Op deze manier worden deze poorten verbonden, waardoor het netwerkverkeer die op de gedefinieerde poort van de host toekomt, doorgezonden wordt naar de verbonden poort op de container. Een best practice hierbij is om altijd de volledige connectie te definiëren, dit wil zeggen dat zowel de de externe interface, de externe host poort als de interne container poort meegeven wordt in het run commando. Hierbij worden op de container best enkel poorten geopend die effectief gebruikt worden door deze container. Dit zorgt ervoor dat het aanvalsvlak via het netwerk op deze container verkleint. TCP/IP poorten onder 1024 op de host machine worden best ook vermeden om op te binden. Uiteraard zijn er services die hier op moeten binden, zoals een HTTP-proxy die op poort 80 moet luisteren om te kunnen functioneren, die ook veiliger als ze hierop gebonden zijn. Een goede maatstaf om te weten opdat een container op een geprivilegieerde poort gebonden moet worden, is kijken indien men de service die in de container draait, ook wanneer die op een virtuele machine draait, op een poort onder 1024 zou binden. Daarnaast hebben we net zoals bij Kernel Capabilities ook bij networking een manier om het intern Docker netwerk te omzeilen. Een container gebruikt standaard een bridged netwerk. Hierdoor wordt deze container in een aparte netwerk stack gestoken. Als we de optie ´--net=host´ meegeven overschrijven we dit en geven we de container toegang tot alle netwerkinterfaces van de host machine. Dit moet, tenzij in zeer specifieke toepassingen, niet gebruikt worden.


%TODO Mounting van directories
\subsection{Container configuratie}

Om extra beveiliging te voorzien voor de Docker containers is het mogelijk (indien ondersteund door de host besturingssysteem) om AppArmor en SeLinux te gebruiken. Met AppArmor is het mogelijk om voor elke container apart een AppArmor profiel te maken. SeLinux voor Docker moet aangezet worden bij het opstarten van de Docker Daemon. Dit kan simpel gedaan worden door de Daemon bij het starten de optie ´--selinux-enabled´ mee te geven. Hierna kunnen we SeLinux security opties meegeven aan containers. Zowel AppArmor en Selinux zorgen ervoor dat we elke container individueel kunnen configureren op vlak van security. Het individueel configureren van SeLinux en AppArmor voor containers kan vergeleken worden met het configureren voor processen op een niet-Docker systeem. Dit is buiten de scope van deze Bachelorproef en hier wordt daarom niet verder op ingegaan.

Daarnaast is een element dat we moeten bekijken wat we nu precies in een container willen draaien qua processen. Het idee achter Docker is om één enkele applicatie per container op te delen. Docker luistert standaard enkel op één hoofdproces. Dus indien we meerdere processen willen uitvoeren op één container, die niet gemaakt zijn om vanuit één hoofdproces op te starten en informatie terug door te geven via dit hoofd proces, moeten we procesbeheer gaan bijvoegen in de container. Dit zorgt er onder andere voor dat een container, wat algemeen gezien een simpel iets is, weer zeer complex wordt. Anderzijds zorgt dit ervoor dat je de processen binnen de container niet meer op een optimale manier kunt monitoren met Docker. Een voorbeeld hiervan zou zijn indien zowel de applicatie als de database op eenzelfde container staat. Inden deze configuratie zou gebruikt worden, zal er al snel opgemerkt worden dat het heel wat makkelijker zou zijn om gewoon de database uit de applicatie container te halen en in zijn eigen container op te starten. Dit geeft ons naast een robuuster systeem (als de applicatie crasht zou het kunnen dat dit de database mee doet crashen indien het op dezelfde container draait) ook een meer schaalbaar systeem door gewoon meer applicatiecontainers te linken met dezelfde database container. Een ander voorbeeld van een nutteloos element om in een container te draaien is ssh, containers hebben in de meeste gevallen geen ssh nodig. Alle containers zijn namelijk al toegankelijk via het hostsysteem met het commando ´docker exec CONTAINER´ dus is daar de meest aangewezen plaats om SSH te installeren. Met een installatie van ssh op de hostmachine, kan er met een ssh connectie naar de host makkelijk ingelogd worden op elke container.
