\chapter{Managebility}

Wanneer de managebility van Docker containers onderzocht wordt, zijn er drie opties, in de eerste plaats kan alles manueel gemanaged worden. Daarnaast kan  ook alles automatisch laten gemanaged worden door een proces manager. Ofwel kunnen er tools gebruikt worden om onze containers te beheren. Als men bezig is met het beheren van Docker containers moet er ook rekening gehouden worden met het beheren van onze docker images. 

\section{Manueel beheren}

Een optie waarmee meestal begonnen wordt is het manueel managen van de Docker containers. Bij deze optie zullen de docker containers beheren enkel met de commando's die docker ons aanbied. Elke container invividueel moeten aanroepen in elk commando lijkt mogelijk in het begin. Maar na het manueel uitvoeren van deze commando's wordt het al snel duidelijk datmet behulp van een kort script of een makefile het beheren heel wat makkelijker kan. Op deze manier kunnen we zowel onze containers managen als onze Docker images builden met veel kortere commando's. Containers managen met de volledige docker commando's of scripts/makefiles is doenbaar indien we enkele containers moeten managen zoals in een ontwikkel omgeving of een stabiele kleine omgeving. Het manueel managen van onze containers verwacht ook meer stabiliteit van onze containers. Indien er niemand de containers aan het monitoren zou zijn wanneer een container crasht, moet er gewacht worden tot er iemand deze manueel terug opstart. Hierdoor kan het dus gebeuren dat een container relatief langer onbeschikbaar is. In een development omgeving zijn de vereisten voor uptime veel lager. Hier is het manueel managen van onze containers en images wel een optie. Bij het manueel managen wordt er ook gebruik gemaakt van scripts of makefiles. Een voorbeeld van beide vind u hieronder.

\begin{lstlisting}[language=bash, style=configstyle]
CONTAINER ?= redis
IMAGE_NAME ?= redis
IMAGE_VERSION ?= latest
HOSTNAME ?= redis-host


# full image name
IMAGE = $(IMAGE_NAME):$(IMAGE_VERSION)
# full ssh connect vars
CONNECT = $(REMOTE_USER_NAME)@$(IP)

build:
	docker build -t $(IMAGE) .

run:
	docker run -d --name $(CONTAINER) -h $(HOSTNAME) $(IMAGE)

logs:
	docker logs $(CONTAINER)

kill_container:
	docker kill $(CONTAINER)

remove_container:
	docker rm $(CONTAINER)

remove_image:
	docker rmi $(IMAGE)

update: kill_container remove_container remove_image build run
\end{lstlisting}

Met deze makefile hebben wordt er toegang verkregen tot de verschillende commando's die normaal volledig met de hand zouden moeten uitgeschreven worden, maar nu kunnen deze aangeroepen worden met 'make COMMANDO'. Als we dit bekijken met de voorgaande makefile zien we dat om onze redis container uit te voeren enkel 'make run' moest ingeven worden in het terminal venster terwijl anders 'docker run --name redis redis' moest worden gebruikt. Bij dit voorbeeld is het commando dat zonder makefile zouden moeten uitgevoerd worden nog beperkt maar als er meer opties meegeven moeten worden met het commando kan het interessanter zijn  om een kort make commando uit te voeren dan elke keer dat dezelfde container  opgestart moet worden opnieuw het volledige commando uit te typen. Ook wordt consistentie behouden bij het uitvoeren van de commando's via Makefiles en kunnen eventuele menselijken fouten zoals typfouten makkelijker geëlimineerd worden. Het ander voordeel dat hier ook uit voorkomt is dat make commando's ook kunnen aanroepen worden in andere make commando's, hierdoor kan het commando 'make update' gemaakt worden. Deze verwijdert onze oude bestaande container en image. Daarna wordt een nieuwe image gemaakt en ingeladen in een nieuwe container. Bovenaan de makefile vinden we ook variabelen die gebruikt kunnen worden in onze make commando's. Met behulp van deze variabelen kunnen we een makefile snel aanpassen voor het gebruik met een ander commando.

%TODO insert script here

Bij het gebruik van scripts wordt ongeveer dezelfde functionaliteit verkregen als bij makefiles, enkel is de syntaxis voor het uitvoeren anders en de manier waarop het geschreven is. Hierboven vinden we een script die dezelfde functionaliteit zou hebben als het 'make update' commando.



\section{Managen met process manager}

Een andere optie die er is, zijn de containers managen met een process-manager. Hierbij kunnen de containers bekeken worden alsof het processen zijn. Als de  containers gemanaged worden met een process-manager, moeten configuratie bestanden handmatig geschreven worden voor de gekozen process-manager. Die gekozen manager zal dan deze bestanden inlezen en vervolgens de met de parameters vanuit de configuratie bestanden onze containers starten en managen zoals het een ander proces zou beheren. Een voorbeeld hiervan is het gebruik van Systemd Unit files. Dit is hieronder uitgewerkt voor de Redis Container uit hoofdstuk drie.

%TODO systemd unit files uitleggen

\begin{lstlisting}[language=bash, style=configstyle]
[Unit]
Description=Run a redis container
Author=Toon Lamberigts & Tomas Vercautter
Requires=docker.service

[Service]
ExecStartPre=-/usr/bin/docker rm redis-container
ExecStart=/usr/bin/docker run --rm --name redis-container -h redis-host redis
ExecStop=/usr/bin/docker stop -t 2 redis
\end{lstlisting}

Deze unit file zal bij het proberen starten van de bijbehorende service ingeladen worden. Het 'ExecStartPre' gedeelte wordt uitgevoerd voor de service effectief zal starten. Deze zal de container verwijderen indien deze bestaat, om zeker te zijn dat er geen container genaamd 'redis-container' bestaat. Dit zou een fout genereren bij het opstarten waardoor de service zou falen. Dus uit voorzorg proberen we deze container eerst te verwijderen voor het opstarten. Door het meegeven van een '-' voor het commando, zal het opstarten van deze service niet falen indien er geen container genaamd 'redis-container' bestaat (en het proberen verwijderen ervan een fout teruggeeft). Het onderdeel 'ExecStart' is het effectieve commando dat de service zal uitvoeren en zich aan zal binden. In dit geval is dat onze redis container. We geven bij het opstarten via unit files geen flag '-d', detached mee omdat we willen dat de container zich bind aan de service waardoor we de container kunnen monitoren via de aangemaakte service. Hiervoor geven we ook de '--rm' tag mee, deze zorgt ervoor dat bij het stoppen van de container hij ook direct verwijdert wordt. Dit gebeurt jammer genoeg enkel bij een correcte terminatie van de container. Dus indien de docker service crasht zal deze niet verwijderd worden maar wel gestopt.

\section{Managen met tools}

Een derde optie die bestaat is het managen van de containers met tools. In dit onderdeel wordt er niet uitgebreid beschrijven hoe elke tool werkt en hoe die moet gebruikt worden, dit moet geval per geval bekeken worden als dit een meerwaarde betekend voor het project met docker. In dit hoofdstuk wordt er geprobeert een voorbeeld te geven hoe tools zouden kunnen gebruiken worden in een docker omgeving. De tool waarover wat meer uitleg zal gegeven worden is Vagrant. Deze tool is enkel een voorbeeld en is zeker niet de enige tool. Vagrant is een tool ontwikkeld voor het gebruik in ontwikkelomgevingen en wordt door Hashicorp ontwikkeld. Daarnaast hebben ze ook Nomad, een nieuwe tool die in moet staan voor de deployment van containers. Andere bedrijven hebben andere tools en zoals hierboven vermeld moet voor elke use case gekeken worden hoe tools eventueel kunnen geïntegreerd worden in het ontwikkelproces met docker. Het voorbeeld van Vagrant dient om een idee de vormen van hoe tools gebruikt kunnen worden. Docker bied zelf ook een tool aan voor het managen van docker containers namelijk Docker Compose. Een voorbeeld van een configuratiebestand voor Vagrant en Docker Compose worden hieronder weergegeven. Beide zetten een docker container die redis noemt, op van een image genaamd redis.

\begin{lstlisting}[language=bash, style=configstyle]
Vagrant.configure("2") do |config|
	config.vm.provision "docker" do |d|
		d.image = "redis"
		d.name = "redis"
	end
end
\end{lstlisting}
Dit is een voorbeeld van een Vagrant bestand voor docker. Deze kan opgestart worden met het commando 'vagrant up'


\begin{lstlisting}[language=bash, style=configstyle]
#redis image
redis:
	image: redis
\end{lstlisting}
Dit is een voorbeeld van een Docker-compose bestand voor docker. Deze kan opgestart worden met het commando 'docker-compose up'

