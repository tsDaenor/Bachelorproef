\chapter{Managebility}

Wanneer we de managebility van Docker containers gaan bekijken hebben we drie opties, in de eerste plaats kunnen we alles manueel managen. Daarnaast kunnen we ook alles automatisch laten managen door een proces manager. Ofwel kunnen we tools gebruiken om onze containers te managen. Als we bezig zijn met het managen van Docker containers moeten we ook rekening houden met het managen van onze docker images. 

\section{Manueel managen}

Een optie waarmee de meesten beginnen is het manueel managen van onze Docker containers. Bij Deze optie gaan we Manueel onze docker containers gaan uitvoeren of met behulp van een kort script of een makefile. Op deze manier kunnen we zowel onze containers managen als onze Docker images builden. Op deze manier onze containers managen is doenbaar indien we enkele containers moeten managen zoals in een development omgeving of een stabiele kleine omgeving. Het manueel managen van onze containers verwacht ook meer stabiliteit van onze containers. Indien er niemand de containers aan het monitoren zou zijn wanneer een container crasht, Moet er gewacht worden tot er iemand deze weer herstart. Hierdoor kan het dus gebeuren dat een container relatief langer onbeschikbaar is. In een development omgeving zijn de vereisten voor uptime veel lager. Hier is het manueel managen van onze containers en images wel een optie. Bij het manueel managen wordt er ook gebruik gemaakt van scripts of makefiles. Een voorbeeld van beide vind u hieronder.

\begin{lstlisting}[language=bash, style=configstyle]
CONTAINER ?= redis
IMAGE_NAME ?= redis
IMAGE_VERSION ?= latest
HOSTNAME ?= redis-host


# full image name
IMAGE = $(IMAGE_NAME):$(IMAGE_VERSION)
# full ssh connect vars
CONNECT = $(REMOTE_USER_NAME)@$(IP)

build:
	docker build -t $(IMAGE) .

run:
	docker run -d --name $(CONTAINER) -h $(HOSTNAME) $(IMAGE)

logs:
	docker logs $(CONTAINER)

kill_container:
	docker kill $(CONTAINER)

remove_container:
	docker rm $(CONTAINER)

remove_image:
	docker rmi $(IMAGE)

update: kill_container remove_container remove_image build run
\end{lstlisting}

In deze makefile hebben we toegang tot de verschillende commando's die we normaal volledig met de hand zouden uitschrijven maar nu kunnen we deze aanroepen met 'make COMMANDO'. Als we dit bekijken met de voorgaande makefile zien we dat we voor onze redis container uit te voeren enkel 'make run' moeten uitvoeren terweil we anders 'docker run --name redis redis' moesten uitvoeren. Bij dit voorbeeld is het commando dat we zonder makefile zouden moeten uitvoeren nog beperkt maar als we meer opties meegeven met ons commando wordt het interresanter voor ons om een kort make commando uit te voeren dan elke keer dat we dezelfde container willen starten opnieuw het complete commando uit te typen. Ook behouden we onze consistentie bij het uitvoeren van de commando's en elimineren we eventuele menselijke fouten zoals typfouten. Het voordeel dat we hier hebben is dat we make commando's ook kunnen aanroepen in andere make commando's hierdoor kunnen we de het commando 'make update' maken. Deze verwijdert onze oude bestaande container en image. Daarna wordt een nieuwe image gemaakt en deze ingeladen in een nieuwe container. Bovenaan de makefile vinden we ook variabelen die gebruikt kunnen worden in onze make commando's. Met behulp van deze variabelen kunnen we een makefile snel aanpassen voor het gebruik met een ander commando.

%TODO insert script here

Bij het gebruik van scripts verkrijgen we ongeveer dezelfde functionaliteit als bij makefiles, enkel is de syntaks voor het uitvoeren anders en de manier waarop het geschreven is. Hierboven vinden we een script die dezelfde functionaliteit zou hebben als het 'make update' commando.



\section{Managen met process manager}

Een andere optie die we hebben is onze containers managen met een process manager. Hierbij kunnen we onze containers gaan bekijken alsof het processen zijn. Als we oze containers managen met een process manager gaan we zelf configuratie bestanden gaan schrijven voor de gekozen process manager. Die gekozen manager zal dan deze bestanden inlezen en vervolgens de met de parameters vanuit de configuratie bestanden onze containers starten en managen zoals het een ander process zou beheren. Een voorbeeld hiervan is het gebruik van Systemd Unit files. Dit vinden we hieronder uitgewerkt voor de Redis Container uit hoofdstuk drie.

\begin{lstlisting}[language=bash, style=configstyle]
[Unit]
Description=Run a redis container
Author=Toon Lamberigts & Tomas Vercautter
Requires=docker.service

[Service]
ExecStartPre=-/usr/bin/docker rm redis-container
ExecStart=/usr/bin/docker run --rm --name redis-container -h redis-host redis
ExecStop=/usr/bin/docker stop -t 2 redis
\end{lstlisting}

Deze unit file zal bij het proberen starten van de bij behorende service ingeladen worden. Het 'ExecStartPre' gedeelte wordt uigevoerd voor de service effectief zal starten. Deze zal de container verwijderen indien deze bestaat, dit is om zeker te zijn dat er geen container genaamd 'redis-container' bestaat. Dit zou een fout genereren bij het opstarten van waardoor de service zou falen. Dus uit voorzorg proberen we te verwijderen voor het opstarten. Het onderdeel 'ExecStart' is het effectieve commando die de service zal uitvoeren en zich aan zal binden. In dit geval is dat onze redis container. We geven bij het opstarten via unit files geen flag '-d', detached mee omdat we willen dat de container zich bind aan de service waardoor we de container kunnen monitoren via de aangemaakte service. Hiervoor geven we ook de '--rm' tag mee, deze zorgt ervoor dat bij het stoppen van de container hij ook direct verwijdert wordt. Dit gebeurt jammer genoeg enkel bij een correcte terminatie van de container, dus indien de docker service crasht zal deze niet verwijderd worden maar wel gestopt.

\section{Managen met tools}

Een derde optie die we hebben is het managen met tools. In dit onderdeel zal ik niet uitgebreid beschrijven hoe elke tool werkt en hoe je die moet gebruiken, dit moet geval per geval bekeken worden als dit een meerwaarde betekend voor het project met docker. Ik zal hier proberen een voorbeeld te geven hoe men tools zou kunnen gebruiken in een docker omgeving. De tool waarmee ik wat meer uitleg zal geven is vagrant. Deze tool is enkel een voorbeeld en is zeker niet de enige tool hievoor. Vagrant een tool ontwikkeld voor het gebruik in development en wordt door hashicorp ontwikkeld. Daarnaast hebben ze ook Nomad, een nieuwe tool die in moet staan voor de deployment van containers. Andere bedrijven hebben andere tools en zoals hierboven vermeld moet voor elke use case gekeken worden hoe tools eventueel kunnen geïntegreerd worden in het ontwikkelproces met docker. Het voorbeeld van Vagrant dient om een idee de vormen van hoe tools gebruikt kunnen worden. Docker bied zelf ook een tool aan voor het managen van docker containers namelijk Docker Compost. Een voorbeeld van een configuratiefile voor Vagrant en Docker Compose worden hieronder weergegeven. Beide zetten een docker container genaamd redis op van een image genaamd redis.

\begin{lstlisting}[language=bash, style=configstyle]
Vagrant.configure("2") do |config|
	config.vm.provision "docker" do |d|
		d.image = "redis"
		d.name = "redis"
	end
end
\end{lstlisting}
Dit is een voorbeeld van een Vagrant file voor docker. Deze kan opgestart worden met het commando 'vagrant up'


\begin{lstlisting}[language=bash, style=configstyle]
#redis image
redis:
	image: redis
\end{lstlisting}
Dit is een voorbeeld van een Docker-compose file voor docker. Deze kan opgestart worden met het commando 'docker-compose up'

