\section{Host Configuration}

Een van de belangrijkste onderdelen in het beveiligen van de Docker stack is het beveiligen van de host machine waarop de docker containers zich zullen bevinden. Dit heeft drie onderdelen namelijk het up to date houden van het systeem, het juist configureren van het systeem, en het auditten van het systeem.

\subsection{Up to date houden van het systeem}

Om de veiligheid van een systeem te kunnen garanderen moet er ten allen tijde gezorgd worden dat het systeem up to date blijft. Dit zorgt ervoor dat security flaws die gekend en gepatcht zijn ook op het systeem beveiligd zijn. Doordat we Docker gebruiken komt hier nog een extra laag bij. Bij het Updaten van een machine waarop de services rechtstreeks draaien moet enkel het host systeem, de kernel van het host systeem en de services up to date blijven. Doordat Docker een kernel deelt met het hostsysteem wordt het up to date houden van de kernel belangrijker. Naast het onderhouden van de voorvernoemde elementen moet er nu, doordat we Docker gebruiken, nog met twee extra dingen rekening gehouden worden. Enerzijds moet Docker zelf nu ook upgedate worden, maar een onderdeel die soms vergeten wordt is dat elke container nu ook zijn eigen besturingssysteem gebruikt. Hierover wordt er verder bij het beveiligen van Container images meer uitleg over gegeven.

% TODO dev tools? and explanaition about stable releases

\subsection{Configureren van het systeem}

Bij het configureren van het hostysteem wordt net zoals bij het up to date houden van het systeem dezelfde stappen van een native applicatie op een hostysteem configureren gevolgd. Maar door de extra laag komen daar nog extra stappen bij. Bij een systeem waar Docker op draait wordt er aangeraden om een aparte partitie te creëren voor de containers. Dit is een simpele operatie waardoor alle docker gerelateerde files zich niet meer tussen de files van het hostsysteem bevinden.

Daarnaast moet ook onder controle gehouden worden wie toegang krijgt tot de docker daemon. De Docker daemon heeft 'root' access, gebruikers die worden toegevoegd aan de 'docker' usergroup en geen 'root' privileges hebben kunnen hierdoor 'root' access verkrijgen tot het hostsysteem. Men directories sharen tussen het host systeem en een guest container. Doordat containers standaard altijd runnen als 'root' heeft de container als de '/' directory gemount is op deze container ongelimiteerde toegang tot het volledige host systeem. Hierdoor kunnen zonder ristricties aanpassingen aan het hostysteem gedaan worden. Dit betekend dat een gebruiker die toegang heeft tot de docker daemon verhoogde rechten kan verkrijgen door gewoon een container op te starten.

Zoals bij een virtuele machine waarop de services rechtstreeks wordt ook bij een systeem waar docker containers op draaien aangeraden overbodige services uit te schakelen of te verwijderen. Moest een gebruiker door een van deze services toegang krijgen tot het hostsysteem. Zou hij makkelijker 'root' rechten kunnen vergrijgen volgens de manier beschreven in de vorige paragraaf.

\subsection{Auditen van het systeem}

% TODO find paper on auditing systems
