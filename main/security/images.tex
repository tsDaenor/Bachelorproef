\section{Security of Docker images}

Als men het hebben over de beveiliging van Docker images zijn er twee grote categoriën die moeten onderscheiden worden. Een eerste categorie is wanneer we werken met Docker images die afkomstig zijn van de officiële Docker Hub Registry, een repository waar iedereen zijn images op kan posten en kan distrubueren. Daarnaast hebben we ook de mogelijkheid om onze eigen images te maken.

\subsection{Images van de Docker hub}

Wanneer we Docker images gebruiken van de officiële Docker Hub Registry is een handig hulpmiddel het commando ´docker search zoekterm´. Dit geeft ons een lijst van alle Docker images die voldoen aan deze zoekterm. Een voorbeeld hiervan vindt u hieronder.

%TODO fukcing kut tables

\begin{lstlisting}[language=bash]
$ docker search busybox
\end{lstlisting}
\begin{table}[!ht]
	\scriptsize
	\centering
	\begin{tabular}{lllll}
		NAME                 & DESCRIPTION                                   & STARS & OFFICIAL & AUTOMATED \\
		busybox              & Busybox base image.                           & 316            & [OK]              &                    \\
		progrium/busybox     &                                               & 50             &                   & [OK]               \\
		radial/busyboxplus   & Full-chain, Internet enabled, busybox made... & 8              &                   & [OK]               \\
		odise/busybox-python &                                               & 2              &                   & [OK]              
	\end{tabular}
\end{table}



Een van de dingen die hierbij opvallen is de kolom `OFFICIAL`. Een image die als "official" is aangeduid, is door het goedkeuringsproces van Docker zelf gegaan. Dit houdt onder andere in dat de images onderhouden worden, bij herhaaldelijk builden altijd hetzelfde resultaat bekomen wordt, consistent maar toch duidelijk zijn en goed beveiligd zijn. Meer hierover in het deel over eigen Docker images beveiligen. 

De officiële images zijn in het algemeen veiliger om te gebruiken dan niet officiële images. Dit wil niet zeggen dat onofficiële images niet veilig kunnen zijn, maar deze zijn niet gecontroleerd geweest door Docker zelf. Doordat we bij het downloaden van een Docker image niet direct kunnen zien wat deze zal doen, wordt het dus sterk afgeraden om Docker images van niet vertrouwelijke bronnen te gebruiken.

Van elke image die gevonden wordt met ´docker search´ en dus op de officiële repositories staat (ervan uitgaand dat er manueel geen andere repositories zijn toegevoegd) is online op de Docker hub webpagina een Dockerfile te vinden. Met deze Dockerfile kunnen we ook zelf kijken hoe de image precies is opgebouwd. Dit kan wel snel een tijdrovend proces worden aangezien Docker images kunnen gebaseerd zijn op andere images. Hierdoor kan voorkomen dat er meer dan een paar Dockerfiles moeten bekeken worden voordat men weet wat er precies allemaal inbegrepen zit in de onderzochte image.

Een ander belangrijk element dat met de officiële Docker images probeert opgelost te worden, is dat iedereen zijn eigen manier heeft om de Dockerfiles op te bouwen en software te installeren. Door een methodiek aan te bieden die gevolgd moet worden voordat de image kan opgenomen worden in de officiële Docker hub zorgen ze voor een gelijkaardige opbouw van Dockerfiles bij officiële Docker images.

%TODO not happy vv

De vergelijking met een virtuele machine zonder Docker kan gemaakt worden door de officiële Docker hub te vergelijken met een package archive, waarvan programma's kunnen geïnstalleerd worden in de meeste linux distributies. In beide gevallen kan zowel van de officiële repositories afgehaald worden als zelf niet officiële repositories toegevoegd. Het verschil ligt dan uiteraard in de aangeboden software, Docker images t.o.v. packages. 

\subsection{Zelf Docker Images maken}

Als we zelf Docker images willen maken zijn de eisen die gesteld worden voor officiële Docker images een goede referentie. Uiteraard zijn er hier elementen die voor eigen gebruik niet uiterst noodzakelijk zijn, maar deze vormen een goede basis om van te vertrekken.

Als eerste belangrijk punt moeten we rekening houden met welke packages we willen gebruiken in onze images. Hierbij zijn dezelfde risico's aan verbonden net zoals bij het gebruiken van een virtuele machine zonder Docker. Enkel packages installeren die gebruikt worden en het goed configureren (indien nodig) van packages is hier even belangrijk als bij een gewone virtuele machine. het is daarom aangewezen om bij het installeren van packages de fingerprint te controleren. 

Bij het maken van onze Docker images kunnen we ook bestanden kopiëren in onze container. Dit wordt best zo veel mogelijk vermeden maar indien toch vereist, is het best om de file specifiek te kopiëren en niet een hele map. Bij het copiëren van een volledige map kunnen bij herbuilden ongewenste files in de map bijgekomen zijn, waardoor er een onverwacht of ongewenst resultaat kan bekomen worden. 

Een belangrijk element om bij stil te staan als de gemaakte Docker image robuust wenst gemaakt te worden, is het definiëren van vaste package versies. Hierdoor blijven onze images bij het herbuilden van dezelfde versie altijd hetzelfde. Extra beveiliging kan hierdoor verkregen worden, door het gebruiken van specifieke versies van packages weten we ook altijd exact welke versies van deze packages er in onze image gebruikt worden. Een nadeel hieraan is dat bij het uitkomen van nieuwe updates en eventuele security updates en bugfixes we onze Docker images moeten aanpassen. In het algemeen wordt aangeraden om voorgedefinieerde versies te gebruiken, nieuwe versies van software worden best eerst uitvoerig getest vooraleer in gebruik genomen te worden. Door het definiëren van de versie die we willen gebruiken hebben we hierover volledige controle.

Het maken van onze eigen Docker images kunnen we dus volledig vergelijken met het configureren van een virtuele machine zonder Docker. Hierbij moet er ook gekeken worden naar welke bestanden we willen gebruiken op deze machine alsook welke packages er op geïnstalleerd moeten staan. Dezelfde veiligheidsvoorschriften die gebruikt worden bij het configureren van een virtuele machine zijn dus ook van toepassing op het maken van een eigen Docker image.


