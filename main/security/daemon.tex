\section{Docker daemon configuration}

Bij het beveiligen van de docker stack is ook de configuratie van de docker daemon een belangrijk element. Dit heeft twee grote onderdelen. Enerzijds de configuratie van de docker daemon zelf, dit houd in dat we de daemon zelf zo gaan configureren dat deze zo veilig en robuust mogelijk wordt. Aan de andere kant moeten we ook zorgen dat de configuratie files zelf de juiste file permissies en owners hebben. Hierbij is de vergelijking met een virtuele machine, waar de services rechtstreeks op draaien, moeilijk aangezien er hier geen docker daemon op te vinden is. Omdat het configureren van de Docker daemon dingen verandert doe voor elke container toegepast worden kunnen we het wel vergelijken met hoe de inviduele services globaal geconfigureerd worden.

% TODO find information about docker binding on IP/Port or UNIX socket
% https://docs.docker.com/articles/basics/#bind-docker-to-another-hostport-or-a-unix-socket

\subsection{Globale configuratie}

% TODO find info about Docker switch from lxs to libcontainer

% lxc
% logging level
% storage driver

\subsection{netwerk configuratie}

% inter container communication

% ip tables

% Tls authentication

\subsection{registry configuraion}


\subsection{Permissies voor Docker config files}

Het configureren van de Docker daemon met de juiste instellingen voor zowel networking, registry gebruik en algemene configuratie heeft enkel nut als we ook de configuratie files waarin deze instellingen worden bewaard correct beveiligd zijn. 

% TODO Docker.Service file use?
% TODO DOcker-registry.service file use?
% TODO Docker.socket file use?
% TODO Docker environment file? whut 
% TODO Docker-network invironment file? 
% TODO docker-registry environment file?
% TODO Docker-storage environment file?
% TODO what is stored in /etc/docker folder?
% 

