\chapter{Methodologie}
\label{ch:methodologie}

% TODO: Hoe ben je te werk gegaan? Verdeel je onderzoek in grote fasen, en
% licht in elke fase toe welke stappen je gevolgd hebt. Verantwoord waarom je
% op deze manier te werk gegaan bent. Je moet kunnen aantonen dat je de best
% mogelijke manier toegepast hebt om een antwoord te vinden op de
% onderzoeksvraag.
Een eerste stap in het onderzoek naar problemen zowel op vlak van security als van managebility bij Docker containers was wegwijs geraken in de wereld van Docker. Hiervoor heb is er twee maanden lang elke dag intensief gebruik gemaakt Docker. Hierbij heb is er een volledige omgeving opgezet waarop applicaties gedeployed werden. Hierdoor kon er een beeld verkregen worden van de problemen die andere mensen hebben met Docker en konden ook de eventuele oplossingen getest worden op een omgeving die zo dicht mogelijk aansluit bij een mogelijk productiewaardige omgeving.

Tijdens het opzetten van de omgeving is er ook op zoek gegaan naar eventuele kritieken die op het internet te vinden waren en hoe men deze kritieken staafden. In de eerste plaats is er op zoek gagaan naar eventuele problemen in verband met beveiliging en wanneer die ontstonden. Daarnaast is er onderzoek gedaan naar problemen in verband met monitoring/managebility. Na het intensief gebruiken van Docker en het onderzoek naar problemen was het makkelijker om een beter zicht te vormen over welke problemen er nu effectief bestaan in verband met Docker container en of deze terecht zijn. Maar ook welke problemen er al waren opgelost.

Na het vinden van deze kritieken moet er gekeken worden of deze kritieken ook voorkomen bij virtuele machines. Meerbepaald of er vroeger op dit vlak ook problemen waren met de virtuele oplossingen en indien zo hoe werd dit opgelost? Ook heb ik gekeken naar hoe de beveiliging en managebility van gewone virtuele machines wordt aangepakt. Dit zorgt er voor dat ik een goed idee heb van waar andere systemen problemen mee hadden en hoe dit verholpen werd/word.

Hierbij kan ik dan de gevonden problemen zowel voor managebility als security bij Docker containers gaan vergelijken met de probleempunten bij virtuele machines.  Zijn er elementen die we kunnen oplossen op dezelfde of een gelijkaardige manier als bij virtuele machines. En zijn er dingen die we nog niet hebben tegengekomen bij virtuele machines maar nu wel moeten opgelost worden met Docker?

Als laatste ga ik dan kijken hoe we deze kritiekpunten kunnen verhelpen. Meer bepaald zijn er tools, technieken of best-practices die we kunnen gebruiken om dit te verhelpen of zijn er dingen die we niet zelf kunnen oplossen maar moeten opgelost worden door de Docker defs?